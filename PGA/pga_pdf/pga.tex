% !TeX program = pdflatex

\documentclass[journal]{IEEEtran}
\usepackage{color}
\usepackage{amssymb,amsfonts,amsmath,graphicx,subfigure,algorithm,algorithmicx}
\usepackage[noend]{algpseudocode}

\newtheorem{definition}{Definition}
\renewcommand{\algorithmicrequire}{\textbf{Input:}}
\renewcommand{\algorithmicensure}{\textbf{Output:}}

% *** SPECIALIZED LIST PACKAGES ***
%
%\usepackage{algorithmic}
% algorithmic.sty was written by Peter Williams and Rogerio Brito.
% This package provides an algorithmic environment fo describing algorithms.
% You can use the algorithmic environment in-text or within a figure
% environment to provide for a floating algorithm. Do NOT use the algorithm
% floating environment provided by algorithm.sty (by the same authors) or
% algorithm2e.sty (by Christophe Fiorio) as the IEEE does not use dedicated
% algorithm float types and packages that provide these will not provide
% correct IEEE style captions. The latest version and documentation of
% algorithmic.sty can be obtained at:
% http://www.ctan.org/pkg/algorithms
% Also of interest may be the (relatively newer and more customizable)
% algorithmicx.sty package by Szasz Janos:
% http://www.ctan.org/pkg/algorithmicx



% *** ALIGNMENT PACKAGES ***
%
%\usepackage{array}
% Frank Mittelbach's and David Carlisle's array.sty patches and improves
% the standard LaTeX2e array and tabular environments to provide better
% appearance and additional user controls. As the default LaTeX2e table
% generation code is lacking to the point of almost being broken with
% respect to the quality of the end results, all users are strongly
% advised to use an enhanced (at the very least that provided by array.sty)
% set of table tools. array.sty is already installed on most systems. The
% latest version and documentation can be obtained at:
% http://www.ctan.org/pkg/array


% IEEEtran contains the IEEEeqnarray family of commands that can be used to
% generate multiline equations as well as matrices, tables, etc., of high
% quality.



% *** FLOAT PACKAGES ***
%
%\usepackage{fixltx2e}
% fixltx2e, the successor to the earlier fix2col.sty, was written by
% Frank Mittelbach and David Carlisle. This package corrects a few problems
% in the LaTeX2e kernel, the most notable of which is that in current
% LaTeX2e releases, the ordering of single and double column floats is not
% guaranteed to be preserved. Thus, an unpatched LaTeX2e can allow a
% single column figure to be placed prior to an earlier double column
% figure.
% Be aware that LaTeX2e kernels dated 2015 and later have fixltx2e.sty's
% corrections already built into the system in which case a warning will
% be issued if an attempt is made to load fixltx2e.sty as it is no longer
% needed.
% The latest version and documentation can be found at:
% http://www.ctan.org/pkg/fixltx2e


%\usepackage{stfloats}
% stfloats.sty was written by Sigitas Tolusis. This package gives LaTeX2e
% the ability to do double column floats at the bottom of the page as well
% as the top. (e.g., "\begin{figure*}[!b]" is not normally possible in
% LaTeX2e). It also provides a command:
%\fnbelowfloat
% to enable the placement of footnotes below bottom floats (the standard
% LaTeX2e kernel puts them above bottom floats). This is an invasive package
% which rewrites many portions of the LaTeX2e float routines. It may not work
% with other packages that modify the LaTeX2e float routines. The latest
% version and documentation can be obtained at:
% http://www.ctan.org/pkg/stfloats
% Do not use the stfloats baselinefloat ability as the IEEE does not allow
% \baselineskip to stretch. Authors submitting work to the IEEE should note
% that the IEEE rarely uses double column equations and that authors should try
% to avoid such use. Do not be tempted to use the cuted.sty or midfloat.sty
% packages (also by Sigitas Tolusis) as the IEEE does not format its papers in
% such ways.
% Do not attempt to use stfloats with fixltx2e as they are incompatible.
% Instead, use Morten Hogholm'a dblfloatfix which combines the features
% of both fixltx2e and stfloats:
%
% \usepackage{dblfloatfix}
% The latest version can be found at:
% http://www.ctan.org/pkg/dblfloatfix


%\ifCLASSOPTIONcaptionsoff
%  \usepackage[nomarkers]{endfloat}
% \let\MYoriglatexcaption\caption
% \renewcommand{\caption}[2][\relax]{\MYoriglatexcaption[#2]{#2}}
%\fi
% endfloat.sty was written by James Darrell McCauley, Jeff Goldberg and 
% Axel Sommerfeldt. This package may be useful when used in conjunction with 
% IEEEtran.cls'  captionsoff option. Some IEEE journals/societies require that
% submissions have lists of figures/tables at the end of the paper and that
% figures/tables without any captions are placed on a page by themselves at
% the end of the document. If needed, the draftcls IEEEtran class option or
% \CLASSINPUTbaselinestretch interface can be used to increase the line
% spacing as well. Be sure and use the nomarkers option of endfloat to
% prevent endfloat from "marking" where the figures would have been placed
% in the text. The two hack lines of code above are a slight modification of
% that suggested by in the endfloat docs (section 8.4.1) to ensure that
% the full captions always appear in the list of figures/tables - even if
% the user used the short optional argument of \caption[]{}.
% IEEE papers do not typically make use of \caption[]'s optional argument,
% so this should not be an issue. A similar trick can be used to disable
% captions of packages such as subfig.sty that lack options to turn off
% the subcaptions:
% For subfig.sty:
% \let\MYorigsubfloat\subfloat
% \renewcommand{\subfloat}[2][\relax]{\MYorigsubfloat[]{#2}}
% However, the above trick will not work if both optional arguments of
% the \subfloat command are used. Furthermore, there needs to be a
% description of each subfigure *somewhere* and endfloat does not add
% subfigure captions to its list of figures. Thus, the best approach is to
% avoid the use of subfigure captions (many IEEE journals avoid them anyway)
% and instead reference/explain all the subfigures within the main caption.
% The latest version of endfloat.sty and its documentation can obtained at:
% http://www.ctan.org/pkg/endfloat
%
% The IEEEtran \ifCLASSOPTIONcaptionsoff conditional can also be used
% later in the document, say, to conditionally put the References on a 
% page by themselves.


% correct bad hyphenation here
\hyphenation{op-tical net-works semi-conduc-tor}


\begin{document}
%
% paper title
% Titles are generally capitalized except for words such as a, an, and, as,
% at, but, by, for, in, nor, of, on, or, the, to and up, which are usually
% not capitalized unless they are the first or last word of the title.
% Linebreaks \\ can be used within to get better formatting as desired.
% Do not put math or special symbols in the title.
\title{PGA: Plug-in Inter-layer Graph Alignment for Stable Neighborhoods in Deep Face Recognition}
%
%
% author names and IEEE memberships
% note positions of commas and nonbreaking spaces ( ~ ) LaTeX will not break
% a structure at a ~ so this keeps an author's name from being broken across
% two lines.
% use \thanks{} to gain access to the first footnote area
% a separate \thanks must be used for each paragraph as LaTeX2e's \thanks
% was not built to handle multiple paragraphs
%

\author{Jinkang ZHu
}

% note the % following the last \IEEEmembership and also \thanks - 
% these prevent an unwanted space from occurring between the last author name
% and the end of the author line. i.e., if you had this:
% 
% \author{....lastname \thanks{...} \thanks{...} }
%                     ^------------^------------^----Do not want these spaces!
%
% a space would be appended to the last name and could cause every name on that
% line to be shifted left slightly. This is one of those "LaTeX things". For
% instance, "\textbf{A} \textbf{B}" will typeset as "A B" not "AB". To get
% "AB" then you have to do: "\textbf{A}\textbf{B}"
% \thanks is no different in this regard, so shield the last } of each \thanks
% that ends a line with a % and do not let a space in before the next \thanks.
% Spaces after \IEEEmembership other than the last one are OK (and needed) as
% you are supposed to have spaces between the names. For what it is worth,
% this is a minor point as most people would not even notice if the said evil
% space somehow managed to creep in.



% The paper headers
\markboth{IEEE TRANSACTIONS ON NEURAL NETWORKS AND LEARNING SYSTEMS}%
{IEEE TRANSACTIONS ON NEURAL NETWORKS AND LEARNING SYSTEMS}
% The only time the second header will appear is for the odd numbered pages
% after the title page when using the twoside option.
% 
% *** Note that you probably will NOT want to include the author's ***
% *** name in the headers of peer review papers.                   ***
% You can use \ifCLASSOPTIONpeerreview for conditional compilation here if
% you desire.




% If you want to put a publisher's ID mark on the page you can do it like
% this:
%\IEEEpubid{0000--0000/00\$00.00~\copyright~2015 IEEE}
% Remember, if you use this you must call \IEEEpubidadjcol in the second
% column for its text to clear the IEEEpubid mark.



% make the title area
\maketitle

% As a general rule, do not put math, special symbols or citations
% in the abstract or keywords.
\begin{abstract}
Deep face recognition models often suffer from noisy local geometry (e.g., multi-cluster intra-class structure, unstable hard negatives), which degrades generalization even under strong classification losses such as ArcFace. To stabilize the learned neighborhood structure, we propose a plug-in inter-layer graph alignment (PGA) regularizer that operates within a single network. First, we build a top-k diffusion kernel at multiple intermediate stages and perform K-alignment: the kernel of a lower layer is encouraged to match the EMA-smoothed kernel of the subsequent layer on masked, reliable edges, suppressing transient/erroneous links while reinforcing persistent same-class neighbors. Second, we add a light Z-alignment on projected features to promote directional consistency across layers during early training. PGA is loss-agnostic and complements ArcFace; it introduces negligible training overhead 
O(Bk) and zero inference cost. [[[[introduce experiments ...]]]] and improves neighbor stability and sub-cluster connectivity, with robustness to different batch sampling schemes (e.g., random vs. PK).
\end{abstract}

% Note that keywords are not normally used for peerreview papers.
\begin{IEEEkeywords}
Face Recognition, Inter-Layer Graph Alignment, Self-Distillation, Diffusion Kernel, kNN Graph, EMA Target, ArcFace, Neighborhood Consistency, Regularization.
\end{IEEEkeywords}






% For peer review papers, you can put extra information on the cover
% page as needed:
% \ifCLASSOPTIONpeerreview
% \begin{center} \bfseries EDICS Category: 3-BBND \end{center}
% \fi
%
% For peerreview papers, this IEEEtran command inserts a page break and
% creates the second title. It will be ignored for other modes.
\IEEEpeerreviewmaketitle



\section{Introduction}
\label{sc:intro}
\IEEEPARstart{T}{ime-series} Classification (TSC) is one of the most challenging tasks in data mining \cite{Fawaz:2019}. In recent years, with the remarkable success of deep neural networks (DNNs) in Computer Vision (CV) \cite{Hu:2020, He:2016, Szegedy:2015}, many researchers have tried to employ DNNs in TSC due to the similarity between time-series data (One-dimensional sequence) and image data (Two-dimensional sequence). However, with extensive experiments on various existing DNN-based TSC approaches, we found that DNNs are easy to be overfitting on datasets from the UCR archive \cite{UCRArchive}. To be specific, several experiments are conducted for $3$ typical DNNs, i.e. Fully Convolutional Networks (FCN) \cite{Wang:2017}, Residual Networks (ResNet) \cite{He:2016, Wang:2017}, and InceptionTime \cite{Fawaz:2020}, on the UCR datasets. Selecting InceptionTime as an example, only $40$ datasets have the training and test accuracy gap less than $0.05$, which means the gap on the other $73$ datasets is more than $0.05$, where many of them are more than $0.2$, as shown in Fig. \ref{fg:gap}(a). In addition, Fig. \ref{fg:gap}(b) gives the training and test accuracy with respect to epochs on a specific dataset among UCR datasets, i.e. $Crop$ dataset, where the gap becomes $0.22$ around epoch $100$ and keeps stationary till the end.

\begin{figure}[!t]
	\centering
	\subfigure[The gap between training and test accuracy on $113$ UCR datasets]{
		\includegraphics[width=0.45\columnwidth]{Figure/gap.png}
	}
	\hfil
	\subfigure[The training and test accuracy w.r.t. epochs on $Crop$ dataset]{
		\includegraphics[width=0.45\columnwidth]{Figure/case.png}
	}
	\caption{Two examples showing that InceptionTime, including other DNNs, is easy to be overfitting on UCR datasets}
	\label{fg:gap}
\end{figure}

After thoroughly analyzing the UCR datasets, we claim that the difference between datasets in CV and those in UCR archive contributes the most to the overfitting phenomenon. In detail, we can view this from the perspective of $N$-shot learning, where $N$ represents the training examples per class. In CV, datasets always contain enough training samples for DNNs, e.g. MNIST \cite{LeCun:1998} and CIFAR10 \cite{CIFAR10} are both $5000$-shot learning datasets with $50000$ training samples in total, and ImageNet \cite{Deng:2009} is averagely a $700$-shot learning dataset with more than 14 million training samples. However, in UCR archive, $95$ datasets are less than $100$-shot and $120$ datasets with less than $1000$ training samples, while only $4$ ones with more than $2000$ training samples. For example, \textit{Fungi} is a $1$-shot learning dataset, and \textit{DiatomSizeReduction} contains only $16$ training samples. We conclude this as the few-shot problem in TSC. In summary, solving the few-shot problem is the key to improving accuracy. Interestingly, many state-of-the-art approaches adopted methods for alleviating overfitting without pointing out the overfitting problem, such as the Hierarchical Vote system for Collective Of Transformation-based Ensembles (HIVE-COTE) \cite{Lines:2018} with an ensemble of $37$ classifiers, InceptionTime \cite{Fawaz:2020} with an ensemble of $5$ models, RandOm Convolutional KErnel Transform (ROCKET) \cite{Dempster:2020} with $L2$ regularization and Cross-Validation, etc. Among the aforementioned state-of-the-art methods, ROCKET has the best accuracy and inference time balance in practice.

%In addition, the DNNs in CV is targeted at only one dataset, yet the DNNs in TSC is targeted at more than $100$ UCR datasets. It is hard to ensure the performance of DNNs with one model and one set of hyperparameters, considering the different data distribution and data characteristic of distinct datasets. We conclude this as the multi-dataset problem in TSC.
	
In this paper, instead of employing the ordinary approaches for alleviating overfitting, e.g. $L1$ and $L2$ regularization, Batch Normalization (BN) \cite{Ioffe:2015}, Dropout \cite{Srivastava:2014}, early stopping, etc, we first proposed Label Smoothing based on InceptionTime \cite{Szegedy:2016} (LSTime) for improving the generalization ability of InceptionTime, as soft labels are closer to real-life compared to hard labels. For instance, as shown in Fig. \ref{fg:ls}(a), the true label of that handwritten number is $2$. Yet, it also looks like $3$ intuitively. Thus, giving a hard label $2$ to that number may cause information losses. Also, in stocks, there are many meaningful chart patterns, among which Head\&Shoulders (H\&S), Triple Top (TT), and Double Top (DT) \cite{Wan:2017} are similar. In Fig. \ref{fg:ls}(b), the H\&S chart pattern is close to TT. Therefore, we wish to keep its information of TT. As a consequence, soft labels maintain more information compared to hard labels. However, we would like to get the soft labels automatically rather than to set that manually. Thus, secondly, Knowledge Distillation based on InceptionTime \cite{Hinton:2015} (KDTime) is leveraged to generate the soft labels by a teacher model, which is a pre-trained network with a deep and complex architecture. The predicted labels from the teacher model represent the knowledge learned by it. After that, the knowledge (Predicted soft labels) can be distilled to help the training of a student model, which owns a relatively smaller and simpler architecture. As a consequence, Knowledge Distillation can also reduce the inference time because of the student model. At last, we claim that the teacher is not $100\%$ correct, which means it may produce wrong soft labels and misguide the student. As the ground-truth labels have already been obtained, we propose to simply calibrate the wrong predicted soft labels, in order to maximize the accuracy of InceptionTime, called Knowledge Distillation with Calibration based on InceptionTime (KDCTime). In addition, KDCTime includes two optional calibrating strategies, i.e. KDC by Translating (KDCT) and KDC by Reordering (KDCR).

\begin{figure}[!t]
	\centering
	\subfigure[Soft labels for CV]{
		\includegraphics[width=0.45\columnwidth]{Figure/sla.png}
	}
	\hfil
	\subfigure[Soft labels for TSC]{
		\includegraphics[width=0.45\columnwidth]{Figure/slb.png}
	}
	\caption{Two examples demonstrating the benefits of soft labels}
	\label{fg:ls}
\end{figure}

The InceptionTime model employed in LSTime, KDTime, and KDCTime is a single model instead of $5$ ones, since the training and inference time are essential, which indicates the feasibility of that model. Thus, compared to an ensemble of $5$ models with $6$ Inception modules each in its original version, the InceptionTime model in this paper is only one with $3$ Inception modules. In summary, the main contributions in this paper can be concluded as follows:

\begin{itemize}
	\item Discovered the TSC approaches based on DNNs are normally overfitting on the UCR datasets, which is caused by the few-shot problem of those datasets. Thus, the best direction to improve the performance of DNNs is alleviating overfitting.
	\item Combined Label Smoothing and Knowledge Distillation with InceptionTime respectively, denoted LSTime and KDTime. LSTime trains the InceptionTime model with manually controlled soft labels, while KDTime is able to generate soft labels automatically by the teacher model.
	\item Proposed KDCTime for calibrating incorrect soft labels predicted by the teacher model, where it contains two optional strategies, i.e. KDCT and KDCR. As a consequence, KDCTime further improves the accuracy of KDTime.
	\item We have tested the accuracy, training time, and test time of ROCKET, InceptionTime, LSTime, KDTime, KDCTime. The results show that compared to ROCKET, KDCTime simultaneously improves the accuracy and reduces its inference time with an acceptable training time overhead. In conclusion, the performance of KDCTime is promising.
\end{itemize}

The remainder of this paper is organized as follows: The related work is reviewed in Section \ref{sc:rw}. Next, LSTime, KDTime, and KDCTime are introduced in Section \ref{sc:pa}. The experimental results are discussed in Section \ref{sc:exp}. Finally, Section \ref{sc:conc} concludes the paper.

\section{Related Work}
\label{sc:rw}
TSC, as a traditional time-series mining research direction, has been considered as one of the most challenging problems \cite{Fawaz:2019}. Traditionally, $1$ Nearest Neighbor (1-NN) Classifiers based on Dynamic Time Warping (DTW) distance has been shown to be a very promising approach \cite{Bagnall:2017}. Yet, the time complexity of DTW is unacceptable compared to Euclidean Distance (ED), i.e. $\mathcal{O}(n^2)$ compared to $\mathcal{O}(n)$. Thus, many researchers have tried to accelerate the execution time of DTW. Rakthanmanon et al. \cite{Rakthanmanon:2012} proposed UCR-DTW by leveraging lower bounding and early abandoning. Sakurai et al. \cite{Sakurai:2007} proposed SPRING under the DTW distance, which is able to monitor time-series streams in real-time. Gong et al. \cite{Gong:2018} proposed Forward-Propagation NSPRING (FPNS) for further accelerating the speed of SPRING. Nevertheless, those researches are all concentrating on time complexity. The upper bound of their accuracy is the accuracy of DTW distance.

In order to break through the bottleneck of accuracy, researchers found that the ensembling of several classifiers could significantly improve the accuracy of TSC. Thus, Baydogan et al. \cite{Baydogan:2013} selected an ensemble of decision trees. Kate \cite{Kate:2016} employed an ensemble of several $1$-NN classifiers with different distance measures, including DTW distance. Those methods motivated the development of an ensemble of $35$ classifiers, named Collective Of Transformation-based Ensembles (COTE) \cite{Bagnall:2016}, which ensembles those classifiers over different time-series representations instead of the same ones. After that, Lines et al. \cite{Lines:2018} extended COTE by leveraging a new hierarchical structure with probabilistic voting, called HIVE-COTE, which is currently considered the state-of-the-art approach in terms of accuracy. Nevertheless, in order to achieve such high accuracy, those methods sacrifice the training and inference time, most of which are impractical when datasets are large.

With the rapid development of deep learning, DNNs are widely applied in CV and are also increasingly employed in TSC \cite{Wang:2017}. Cui et al. \cite{Cui:2016} proposed Multi-Scale Convolutional Neural Networks (MCNN), which transforms the time-series into several feature vectors and feeds those vectors into a CNN model. Wang et al. \cite{Wang:2017} implemented several DNN models originated from CV, i.e. MultiLayer Perceptrons (MLP), Fully Convolutional Networks (FCN), and Residual Networks (ResNet), in order to test their performance in TSC, which provides a strong baseline of DNN-based approaches. Based on a more recent DNN structure, i.e. Inception module \cite{Szegedy:2015}, Karimi-Bidhendi et al. \cite{Karimi-Bidhendi:2018} proposed an approach to transform time-series into feature maps using Gramian Angular Difference Field (GADF), and finally feed those maps to an InceptionNet which is pre-trained for image recognition. By extending a more recent version of IncpetionNet, i.e. Inception-V4 \cite{Szegedy:2017}, Fawaz et al. \cite{Fawaz:2020} proposed InceptionTime, which ensembles $5$ Inception-based models to get a promising accuracy. Multi-scale Attention Convolutional Neural Network (MACNN) \cite{Chen:2021} adopted the attention mechanism to further improve the accuracy of MCNN. Instead of utilizing convolutions as parts of the model, RandOm Convolutional KErnel Transform (ROCKET) \cite{Dempster:2020} employed random convolutional kernels as feature extractors converting time-series into feature vectors, which were later fed to a ridge regressor. To the best of the authors' knowledge, ROCKET owns the best accuracy and inference time balance, while other DNN-based methods always require a longer training and inference time. Actually, DNN-based methods always suffer from long training time, e.g. MACNN requires $3$ days of running time for training $85$ datasets from the UCR archive. That builds a huge barrier for researchers to reimplement the approach.

We found that InceptionTime is a quite competitive approach. Using only $1$ InceptionTime model instead of ensembling $5$ ones, it owns a slower yet acceptable training time and a $2$ magnitude less inference time compared to ROCKET. Therefore, when InceptionTime only includes $1$ model instead of ensembling several models, we would like to ensure the accuracy of it by the information obtained from soft labels. The idea of utilizing soft labels was proposed by Szegedy et al. \cite{Szegedy:2016}, which controls the smooth level of soft labels by manually setting a parameter $\varepsilon$. Yet, a better way to determine the smooth level of soft labels is generating soft labels automatically by a teacher model, and training a student model with those labels, the idea of which comes from Knowledge Distillation (KD) \cite{Hinton:2015}. Since then, many extensions of KD were proposed \cite{Gou:2021}. Some researches \cite{Romero:2014, Yim:2017} concentrated on letting the student model learn the feature maps, instead of soft labels, of the teacher model. In addition, Svitov et al. \cite{Svitov:2020} leveraged the predicted labels by the teacher model as class centers, instead of soft labels, guiding the training of the student model. Oki et al. \cite{Oki:2020} integrated KD into triplet loss and utilized the predicted labels as anchor points for guiding the training of the student model. In this paper, instead of employing other types of KD methods, we still concentrated on label-based KD approaches in order to save more execution time. At last, inspired by \cite{Cho:2019, Mirzadeh:2020}, the gap between the student model and the teacher model should be small, or the student model would hard to mimic the teacher model. Therefore, in this paper, a $3$-layer student InceptionTime model is selected as the student model, and a $6$-layer teacher model is employed as the teacher model. 

\section{Proposed Approaches}
\label{sc:pa}
In this section, instead of concentrating on the model, all the proposed approaches are essentially centered on loss functions and labels. First, notations and definitions are given in Section \ref{ssc:nad}. After that, InceptionTime is briefly introduced in Section \ref{ssc:it}. Then, Label Smoothing for InceptionTime (LSTime) is demonstrated in Section \ref{ssc:lsit}. Next, Knowledge Distillation for InceptionTime (KDTime) is depicted in Section \ref{ssc:kdit}. At last, Knowledge Distillation with Calibration for InceptionTime (KDCTime) is illustrated in Section \ref{ssc:kdcit}, where it contains $2$ strategies, i.e. Calibration by Translating (CT) and Calibration by Reordering (CR).

\subsection{Notations and Definitions}
\label{ssc:nad}
\begin{definition}
	A time-series $\mathbf{x}\in\mathbb{R}^{N}$ is defined as a vector, where $x_{i}$ represents the $i$-th value of $\mathbf{x}$. The corresponding class of $\mathbf{x}$ is a scalar $c\in\{1,2,\dots,C\}$, with $C$ classes in total.
\end{definition}

\begin{definition}
	A class label of $\mathbf{x}$ is defined as a vector $\mathbf{y}=\{y_{1}, y_{2},\dots, y_{C}\}$, where $y_{i}\in[0,1]$ represents the probability of $\mathbf{x}$ belonging to class $c$. In addition, Eq. \eqref{eq:sumy} always holds for any $\mathbf{y}$.
\end{definition}

\begin{equation}
	\label{eq:sumy}
	\sum_{i=1}^{C}y_{i}=1
\end{equation}

\begin{definition}
	The true label of $\mathbf{x}$ is defined as an one-hot vector $\mathbf{y}^{h}$, where all $y^{h}_{i}=0$ except $y^{h}_{c}=1$, called the hard label. The equation of $\mathbf{y}^{h}$ is given in Eq. \eqref{eq:y}.
\end{definition}

\begin{equation}
\label{eq:y}
	y^{h}_{i}=
	\begin{cases}
		1,\hfill \text{if}~i=c\\
		0,\hfill \text{if}~i\neq c
	\end{cases}
\end{equation}

\begin{definition}
	A dataset $\mathbf{D}$ is a pair of sets, including a set of time-series $\mathbf{X}=\{\mathbf{x}_{1}, \mathbf{x}_{2},\dots, \mathbf{x}_{M}\}$ and a set of true labels $\mathbf{Y}^{h}=\{\mathbf{y}^{h}_{1}, \mathbf{y}^{h}_{2},\dots, \mathbf{y}^{h}_{M}\}$ respectively, where each time-series $\mathbf{x}_{i}$ corresponds to a true label $\mathbf{y}^{h}_{i}$.
\end{definition}

\begin{definition}
	An InceptionTime model is treated as a function $\mathcal{F}\in\mathbb{F}$ mapping an input $\mathbf{x}$ into an output $\mathbf{z}\in\mathbb{R}^{C}$, where $\mathbb{F}$ represents the hypothesis space, i.e. the space containing all possibilities of $\mathcal{F}$.
\end{definition}

\begin{definition}
	The predicted label $\mathbf{\hat{y}}$ is produced by normalizing $\mathbf{z}$ with the Softmax function (Eq. \eqref{eq:sm}). Thus, Eq. \eqref{eq:sumy} always holds.
\end{definition}

\begin{equation}
\label{eq:sm}
	\sigma(z_{i})=\hat{y}_{i}=\frac{e^{z_{i}}}{\sum_{j=1}^{C}e^{z_{j}}}
\end{equation}

\begin{definition}
	A loss function $\mathcal{L}$ is a function measuring the difference between the predicted label $\mathbf{\hat{y}}$ and the true label $\mathbf{y}^{h}$, in order to determine the performance of $\mathcal{F}$.
\end{definition}

\begin{definition}
	The problem in the paper is defined as follows: Given a dataset $\mathbf{D}$, find an $\mathcal{F}^{\star}$ minimizing the predefined $\mathcal{L}$. Formally, it is demonstrated in Eq. \eqref{eq:prob}.
\end{definition}

\begin{equation}
\label{eq:prob}
	\mathcal{F}^{\star}=\mathop{\arg\min}_{\mathcal{F}\in\mathbb{F}}\sum_{i=1}^{M}\mathcal{L}(\mathbf{y}^{h}_{i},\sigma(\mathcal{F}(\mathbf{x}_{i})))
\end{equation}

To this end, important notations are briefly summarized in Table \ref{tb:NaD}.

%To this end, we claim that the main contributions in the paper are essentially concentrated in loss functions.
	
\begin{table}[!t]
	\centering
	\caption{Notations and definitions}
	\label{tb:NaD}
	\begin{tabular}{cc}
		\hline
		\textbf{Notations} & \textbf{Definitions} \\
		\hline
		$\mathbf{x}$ & A time-series \\
		$\mathbf{y}^{h}$ & The true label w.r.t. $\mathbf{x}$ \\
		$\mathbf{D}$ & A dataset \\
		$\mathbf{X}$ & A set of time-series in $\mathbf{D}$ \\
		$\mathbf{Y}$ & A set of labels in $\mathbf{D}$ \\
		$\mathcal{F}$ & An InceptionTime model \\
		$\mathbf{z}$ & The output of $\mathcal{F}(\mathbf{x})$ \\
		$\sigma$ & The Softmax function \\
		$\mathbf{\hat{y}}$ & The predicted label of $\sigma(\mathbf{z})$ \\
		$\mathcal{L}$ & A loss function \\
		\hline
	\end{tabular}
\end{table}

\subsection{InceptionTime}
\label{ssc:it}
The ordinary Softmax Cross-Entropy Loss is adopted in InceptionTime \cite{Fawaz:2020}. We first implemented the one model version of InceptionTime with $3$ Inception modules, denoted $\mathcal{F}_{S}$. Thus, the loss function of $\mathbf{\hat{y}}=\sigma(\mathcal{F}_{S}(\mathbf{x}))$ is given in Eq. \eqref{eq:sce}.

\begin{equation}
\label{eq:sce}
	\mathcal{L}_{CE}(\mathbf{y}^{h},\mathbf{\hat{y}})=-\sum_{i=1}^{C}y^{h}_{i}\log\hat{y}_{i}
\end{equation}
where it is easy to know the final loss $\mathcal{L}_{CE}$ is only related to $y^{h}_{c}$, as all the other $y^{h}_{i}$ are $0$ (Eq. \eqref{eq:y}). In other words, only the result of $\log\hat{y}_{c}$ is survived in the summation. Therefore, for simplicity, Eq. \eqref{eq:sce} can also be written as Eq. \eqref{eq:sces}.
	
\begin{equation}
\label{eq:sces}
	\mathcal{L}_{CE}(\mathbf{y}^{h},\mathbf{\hat{y}})=-\log\hat{y}_{c}
\end{equation}
	
Note that Eq. \eqref{eq:sces} is the reason that one-hot class label $\mathbf{y}$ is called hard label, as it explicitly selects only the probability of $\mathbf{x}$ belonging class $c$, yet ignoring all other probabilities in the loss function. However, in a more realistic scenario, we believe such a deterministic case is rare. Hence, as also introduced in Section \ref{sc:intro}, a soft version of $\mathbf{y}^{h}$ is more feasible in this case.

\subsection{Label Smoothing for InceptionTime}
\label{ssc:lsit}
Second, following by the assumption in Section \ref{ssc:it}, we implemented Softmax Cross-Entropy with Label Smoothing (LS) \cite{Szegedy:2016}, with $\mathcal{F}_{S}$ as the model. Therefore, the equation of label smoothed $\mathbf{y}^{h}$ is given in Eq. \eqref{eq:yls}, denoted $\mathbf{y}^{l}$.

\begin{equation}
\label{eq:yls}
	y^{l}_{i}=
	\begin{cases}
		(1-\varepsilon)+\varepsilon/C,\hfill \text{if}~i=c\\
		\varepsilon/C,\hfill \text{if}~i\neq c
	\end{cases}
\end{equation}
where $\varepsilon$ is the smoothing coefficient set by users, representing how much the label is smoothed. Note after LS, $y^{l}_{i}$ still satisfies Eq. \eqref{eq:sumy}. Alternatively, Eq. \eqref{eq:yls} can also be written as Eq. \eqref{eq:ylsa}.
	
\begin{equation}
\label{eq:ylsa}
	y^{l}_{i}=(1-\varepsilon)y^{h}_{i}+\frac{\varepsilon}{C}
\end{equation}

As a consequence, the Softmax Cross-Entropy loss with LS is given in Eq. \eqref{eq:scels}.
	
\begin{equation}
\label{eq:scels}
	\begin{aligned}
		\mathcal{L}_{LS}(\mathbf{y}^{l},\mathbf{\hat{y}})
		&=-\sum_{i=1}^{C}y^{l}_{i}\log\hat{y}_{i}\\
		&=-\sum_{i=1}^{C}[(1-\varepsilon)y^{h}_{i}+\frac{\varepsilon}{C}]\log\hat{y}_{i}\\
		&=-\sum_{i=1}^{C}(1-\varepsilon)y^{h}_{i}\log\hat{y}_{i}-\sum_{i=1}^{C}\frac{\varepsilon}{C}\log\hat{y}_{i}\\
		&=(1-\varepsilon)\mathcal{L}_{CE}(\mathbf{y}^{h},\mathbf{\hat{y}})+\varepsilon(-\frac{1}{C}\sum_{i=1}^{C}\log\hat{y}_{i})
	\end{aligned}
\end{equation}
where the left part $(1-\varepsilon)\mathcal{L}(\mathbf{y}^{h},\mathbf{\hat{y}})$ represents the loss from hard labels, while the right part $\varepsilon(-\frac{1}{C}\sum_{i=1}^{C}\log\hat{y}_{i})$ means the loss from soft labels. The smoothing coefficient $\varepsilon$ controls the weights of losses from hard labels and soft labels.
	
Yet, manually controlling the smoothed level of labels by $\varepsilon$ is not the best solution, since, except $y^{l}_{c}$, every smoothed label has the same value. Similar to hard labels, this kind of manually controlled soft labels are not practical in the real world. Therefore, generating flexible soft labels by Knowledge Distillation is then proposed.

\subsection{Knowledge Distillation for InceptionTime}
\label{ssc:kdit}
Third, we implemented Knowledge Distillation (KD) to help us generate soft labels in replacement of manually controlling them. Instead of manually setting up the soft labels, Hinton et al. \cite{Hinton:2015} proposed KD for generating soft labels by a teacher model, which owns a cumbersome architecture with a large number of parameters. Intuitively, the teacher model has more potential to capture the knowledge from training data because of its scale. Next, the predicted labels from the teacher model can be regarded as the knowledge learned by it, denoted $\mathbf{y}^{t}$. Thus, $\mathbf{y}^{t}$ is an automatic soft label compared to the manual soft label $\mathbf{y}^{l}$. Note the one model version of InceptionTime with $6$ Inception modules is incorporated as the teacher model, denoted $\mathcal{F}_{T}$.
	
In addition, instead of directly using Softmax Cross-Entropy loss, Softmax with a Temperature $\tau$ and Kullback-Leibler (KL) divergence loss are adopted. The equation of Softmax with $\tau$ is given in Eq. \eqref{eq:smt}.
	
\begin{equation}
\label{eq:smt}
	y^{\tau}_{i}=\frac{e^{z_{i}/\tau}}{\sum_{j=1}^{C}e^{z_{j}/\tau}}
\end{equation}
where the Temperature $\tau$ is a parameter for fine-tuning the smoothed level of predicted labels $\mathbf{y}^{t}$ from the teacher model and $\mathbf{\hat{y}}$ from the student model, denoted $\mathbf{y}^{t_{\tau}}$ and $\mathbf{\hat{y}}^{\tau}$. Note $\tau=1$ means the labels keep unchanged, while $\tau<1$ or $\tau>1$ represents the labels are steeper or smoother respectively. For an extreme example, if $\tau\to\infty$, we will have $\forall i, y^{\tau}_{i}=1/C$. To this end, the loss of $\mathbf{y}^{t_{\tau}}$ and $\mathbf{\hat{y}}^{\tau}$ can be measured by KL divergence, the equation of which is given in Eq. \eqref{eq:kld}.
	
\begin{equation}
\label{eq:kld}
	\mathcal{L}_{KL}(\mathbf{y}^{t_{\tau}},\mathbf{\hat{y}}^{\tau})=\sum_{i=1}^{C}y_{i}^{t_{\tau}}\log\frac{y_{i}^{t_{\tau}}}{\hat{y}_{i}^{\tau}}
\end{equation}
	
After that, $\mathcal{L}_{KL}(\mathbf{y}^{t_{\tau}},\mathbf{\hat{y}}^{\tau})$ representing the loss of soft labels and $\mathcal{L}_{CE}(\mathbf{y}^{h},\mathbf{\hat{y}})$ representing the loss of hard labels are incorporated into a whole for training the student model, which has a relatively small architecture with less parameters. This procedure is regarded as distilling the knowledge from a teacher model into a student model, called KD, in order to preserve the accuracy of the teacher model while reducing its time and space complexity. Note $\mathcal{F}_{S}$ is selected as the student model, which is the one model version of InceptionTime with $3$ Inception modules. The equation of KD loss is given in Eq. \eqref{eq:kdsce}.
	
\begin{equation}
\label{eq:kdsce}
	\begin{aligned}
		&\mathcal{L}_{KD}(\mathbf{y}^{h},\mathbf{\hat{y}},\mathbf{y}^{t_{\tau}},\mathbf{\hat{y}}^{\tau})\\
		&=(1-\varepsilon)\mathcal{L}_{CE}(\mathbf{y}^{h},\mathbf{\hat{y}})+\varepsilon\tau^2\mathcal{L}_{KL}(\mathbf{y}^{t_{\tau}},\mathbf{\hat{y}}^{\tau})
	\end{aligned}
\end{equation}
where $\varepsilon$ controls the weight of $\mathcal{L}_{CE}(\mathbf{y}^{h},\mathbf{\hat{y}})$ (Eq. \eqref{eq:sce}) and $\mathcal{L}_{KL}(\mathbf{y}^{t_{\tau}},\mathbf{\hat{y}}^{\tau})$ (Eq. \eqref{eq:kld}). Note $\tau^{2}$ is necessary because the scale of $\mathcal{L}_{KL}(\mathbf{y}^{t_{\tau}},\mathbf{\hat{y}}^{\tau})$ becomes smaller after fine-tuning by $\tau$. Thus, multiplying a $\tau^{2}$ helps it to be the same scale of $\mathcal{L}_{CE}(\mathbf{y}^{h},\mathbf{\hat{y}})$, so that the total loss has no preference to $\mathcal{L}_{CE}(\mathbf{y}^{h},\mathbf{\hat{y}})$.
	
Nevertheless, similar to teachers in real life, the teacher model is not ensured to be $100\%$ correct. Sometimes it may misguide the student model to wrong answers. To be specific, the incorrect soft labels generated by the teacher model will also result in the wrong labels predicted by the student model. In order to alleviate the affection of incorrect labels, KD adopts $\mathcal{L}_{CE}(\mathbf{y}^{h},\mathbf{\hat{y}})$ and $\tau$. Nonetheless, that brings additional hyperparameters into the model. Therefore, we would like to propose a better method to alleviate the affection of incorrect labels while not bringing additional hyperparameters.

\subsection{Knowledge Distillation with Calibration for InceptionTime}
\label{ssc:kdcit}
At last, we propose Knowledge Distillation with Calibration (KDC) to calibrate the incorrect soft labels generated by the teacher model before distillation. Note the teacher model and student model are $\mathcal{F}_{T}$ and $\mathcal{F}_{S}$ respectively (Section \ref{ssc:kdit}). In this way, it is not necessary to employ $\mathcal{L}_{CE}(\mathbf{y},\mathbf{\hat{y}})$ and $\tau$ in KDC. In order to calibrate the incorrect soft labels, all labels $\mathbf{y}$ are regarded as vectors geometrically, including the hard label $\mathbf{y}^{h}$ and the soft label $\mathbf{y}^{t}$ generated by the teacher model. From this point of view, according to Eq. \eqref{eq:sumy}, the feasible solution space of $\mathbf{y}$ is a triangular hyperplane, named the label space. In other words, all $\mathbf{y}$ are located on a triangular hyperplane. Fig. \ref{fg:spc}(a) gives an example when $C=2$. In this case, the triangular hyperplane is an $1$-D line in $2$-D space. Next, Fig. \ref{fg:spc}(b) shows another example when $C=3$. In this case, the triangular hyperplane is a $2$-D regular triangle in $3$-D space. Last, the triangular hyperplane is a $3$-D regular tetrahedron in $4$-D space when $C=4$. Nonetheless, we failed to plot a $4$-D space in figures. Note that distinct colors represent the areas of distinct classes. Thus, it is potential to calibrate $\mathbf{y}^{t}$ from its original position to the target position $\mathbf{y}^{h}$, if $\mathbf{y}^{t}$ is located in the wrong area. In addition, all $\mathbf{y}^{h}$ are located at the vertices of the triangular hyperplane, as marked in Fig. \ref{fg:spc}(a) and Fig. \ref{fg:spc}(b).

\begin{figure}[!t]
	\centering
	\subfigure[The $2$-D label space when $C=2$]{
		\includegraphics[width=0.4\columnwidth]{Figure/spc2.png}
	}
	\hfil
	\subfigure[The $3$-D label space when $C=3$]{
		\includegraphics[width=0.4\columnwidth]{Figure/spc3.png}
	}
	\caption{Two examples showing the label spaces when $C=2$ and $C=3$}
	\label{fg:spc}
\end{figure}

Therefore, our main task is to propose a proper method in order to modify $\mathbf{y}^{t}$ to its correct area while its new position is located between $\mathbf{y}^{t}$ and $\mathbf{y}^{h}$. The calibrated $\mathbf{y}^{t}$ is denoted as $\mathbf{y}^{t_{\zeta}}$. To this end, two approaches for calibration are proposed, which are calibration by translating and calibration by reordering. Note that only incorrect predicted labels will be calibrated. Formally, given a $\mathbf{y}^{t}$ and its corresponding $\mathbf{y}^{h}$, $\mathbf{y}^{t_{\zeta}}$ will be computed only when $\mathop{\arg\max}_{i}\{y^{t}_{i}\}\neq\mathop{\arg\max}_{i}\{y^{h}_{i}\}$. In other words, $\mathop{\arg\max}_{i}\{y^{t}_{i}\}\neq c$.

\subsubsection{Calibration by Translating}
\label{sssc:cbt}
Calibration by Translating (CT) represents geometrically translate $\mathbf{y}^{t}$ from its original position to $\mathbf{y}^{h}$, the equation of which is shown in Eq. \eqref{eq:ctl}.

\begin{equation}
\label{eq:ctl}
	\mathbf{y}^{t_{\zeta}}=\mathbf{y}^{t}+\omega(\mathbf{y}^{h}-\mathbf{y}^{t})
\end{equation}
where $(\mathbf{y}^{h}-\mathbf{y}^{t})$ represents the vector from $\mathbf{y}^{t}$ to $\mathbf{y}^{h}$, while $\omega\in[0,1]$ is a calibration coefficient controlling the distance $\mathbf{y}^{t}$ moves towards $\mathbf{y}^{h}$. It is easy to know that $\mathbf{y}^{t_{\zeta}}=\mathbf{y}^{h}$ when $\omega=1$, and $\mathbf{y}^{t_{\zeta}}=\mathbf{y}^{t}$ when $\omega=0$. In this case, it is simply substitution instead of calibration.

Hence, $\omega$ is the key coefficient defining the degree of calibration. We define the equation of $\omega$ as Eq. \eqref{eq:w}.

\begin{equation}
\label{eq:w}
	\omega=\frac{\delta}{\Vert\mathbf{y}^{h}-\mathbf{y}^{t}\Vert_{2}}
\end{equation}
where $\delta$ is the minimum distance between $\mathbf{y}^{h}$ and $\mathbf{y}^{t}$ when $\mathop{\arg\max}_{i}\{y^{t}_{i}\}\neq\mathop{\arg\max}_{i}\{y^{h}_{i}\}$, and $\Vert\mathbf{y}^{h}-\mathbf{y}^{t}\Vert_{2}$ is the current distance between $\mathbf{y}^{h}$ and $\mathbf{y}^{t}$. It is easy to know that $\Vert\mathbf{y}^{h}-\mathbf{y}^{t}\Vert_{2}\geq\delta$ always holds.

To this end, we calculated $\delta$, and got the magic number $\delta=1/\sqrt{2}$. The procedure of calculation is given in Appendix \ref{ap:pcd}. As a consequence, Eq. \eqref{eq:ctl} can be also rewritten as \eqref{eq:ctlf}.

\begin{equation}
\label{eq:ctlf}
	\mathbf{y}^{t_{\zeta}}=\mathbf{y}^{t}+\frac{1}{\sqrt{2}}\frac{\mathbf{y}^{h}-\mathbf{y}^{t}}{\Vert\mathbf{y}^{h}-\mathbf{y}^{t}\Vert_{2}}
\end{equation}
where we know the unit vector $(\mathbf{y}^{h}-\mathbf{y}^{t})/\Vert\mathbf{y}^{h}-\mathbf{y}^{t}\Vert_{2}$ decides the direction of translating, while $\delta=1/\sqrt{2}$ determines the distance of translating. In this way, it ensures that all $\mathbf{y}^{t_{\zeta}}$ stay in the label space, since it guarantees $\omega\leq1$. In addition, it also guarantees that $\omega\geq0$, which leads to a consequence that $\mathbf{y}^{t}$ will not stay unchanged.

%where $\gamma\in[0,1]$ is a coefficient \colorbox{yellow}{controlling }. In addition, $\mathbf{m}=\{m_{1},m_{2},\dots,m_{C}\}$ represents the perfect smooth label, where $\forall m_{i}$, $m_{i}=1/C$. Physically, $\mathbf{m}$ is the center of mass in the triangular hyperplane. Thus, $\mathbf{y}-\mathbf{m}$ is exactly the threshold depicting that $\mathbf{\widetilde{y}}$ and $\mathbf{y}$ are of the same class if $\Vert\mathbf{y}-\mathbf{\widetilde{y}}\Vert_{2}< \Vert\mathbf{y}-\mathbf{m}\Vert_{2}$, otherwise $\mathbf{\widetilde{y}}$ and $\mathbf{y}$ belong to different classes. Note only incorrect predicted labels will be calibrated, which means $\mathbf{\widetilde{y}}$ and $\mathbf{y}$ are always of different classes. Thus, $\Vert\mathbf{y}-\mathbf{m}\Vert_{2}\le \Vert\mathbf{y}-\mathbf{\widetilde{y}}\Vert_{2}$ always holds.
%
%Further, for $\mathbf{y}$, we know only $y_{c}=1$ and all other $y_{i}=0$ according to Eq. \eqref{eq:y}. Hence, $\Vert\mathbf{y}-\mathbf{m}\Vert_{2}$ satisfies the Eq. \eqref{eq:y-m}. Note no matter which class $\mathbf{y}$ belongs to, Eq. \eqref{eq:y-m} always holds.
%	
%\begin{equation}
%\label{eq:y-m}
%	\begin{aligned}
%		\Vert\mathbf{y}-\mathbf{m}\Vert_{2}
%		&=\sqrt{(1-\frac{1}{C})^{2}+(C-1)(-\frac{1}{C})^{2}}\\
%		&=\sqrt{\frac{(C-1)^{2}}{C^{2}}+\frac{C-1}{C^{2}}}\\
%		&=\sqrt{\frac{C-1}{C}}
%	\end{aligned}
%\end{equation}
%	
%Based on Eq. \eqref{eq:ctl}, Eq. \eqref{eq:w} and Eq. \eqref{eq:y-m}, the equation of $\mathbf{\widetilde{y}}^{\zeta}$ can be rewritten as Eq. \eqref{eq:y^c}.
%\begin{equation}
%\label{eq:y^c}
%	\mathbf{\widetilde{y}}^{\zeta}=\mathbf{\widetilde{y}}+\gamma\sqrt{\frac{C-1}{C}}\frac{\mathbf{y}-\mathbf{\widetilde{y}}}{\Vert\mathbf{y}-\mathbf{\widetilde{y}}\Vert_{2}}
%\end{equation}
%
%From Eq. \eqref{eq:y^c}, we know that $\mathbf{\widetilde{y}}$ is the incorrect predicted label waiting for calibration. The direction of calibration is determined by the unit vector $(\mathbf{y}-\mathbf{\widetilde{y}})/\Vert\mathbf{y}-\mathbf{\widetilde{y}}\Vert_{2}$. The distance of calibration is defined by $\gamma\sqrt{(C-1)/C}$. Finally, $\mathbf{\widetilde{y}}^{\zeta}$ is the calibrated label. Note that $\gamma\sqrt{(C-1)/C}$ ensures $\mathbf{\widetilde{y}}$ will not be simply replaced by the hard label $\mathbf{y}$.

\subsubsection{Calibration by Reordering}
\label{sssc:cbr}
Calibration by Reordering (CR) represents reprioritizing the values of the incorrect predicted label based on a specific strategy. To be concrete, given a $\mathbf{y}^{t}$ and its corresponding $\mathbf{y}^{h}$, some $y^{t}_{i}$ will be resorted if $\mathop{\arg\max}_{i}\{y^{t}_{i}\}\neq\mathop{\arg\max}_{i}\{y^{h}_{i}\}$. Therefore, our main task is to design a reordering strategy.

Given a $\mathbf{y}^{t}$ awaiting for reordering. The reordering strategy is designed as follows: 1) $\mathbf{y}^{t}$ is sorted by descending order. The sorted $\mathbf{y}^{t}$ is denoted $\mathbf{y}^{t_{s}}$. Thus, since the descending order of $y^{t}_{i}$ is random, we have $\mathbf{y}^{t_{s}}=\{y^{t_{s}}_{1}, y^{t_{s}}_{2},\dots, y^{t_{s}}_{C}\}=\{y^{t}_{1st},y^{t}_{2nd},\dots\,y^{t}_{Cth}\}$, where $y^{t}_{1st}$ is the largest $y^{t}_{i}$, $y^{t}_{2nd}$ is the second largest $y^{t}_{i}$, and so on and so forth. Note $\mathbf{y}^{t_{s}}$ is in descending order, which means $y^{t_{s}}_{1}=y^{t}_{1st}=\max_{i}\{y^{t}_{i}\}$. 2) After defining a temporary value $y^{t}_{tmp}=y^{t}_{1st}$, the value of $y^{t}_{(i+1)th}$ will be assigned to $y^{t}_{ith}$, from $y^{t}_{1st}$, $y^{t}_{2nd}$, all the way to $y^{t}_{ith}=y^{t}_{c}$. 3) Assign the value of $y^{t}_{tmp}$ to $y^{t}_{c}$.

The algorithm of the whole procedure is given in Algorithm \ref{ag:cr}. It ensures that $\mathbf{y}^{t_{\zeta}}$ is located in the label space, since $\mathbf{y}^{t_{\zeta}}$ is only the reordered version of $\mathbf{y}^{t}$. In addition, it guarantees that $\mathbf{y}^{t_{\zeta}}$ is geometrically located
in its class area. In other words, $\mathop{\arg\max}_{i}\{y^{t_{\zeta}}_{i}\}=\mathop{\arg\max}_{i}\{y^{h}_{i}\}$. As a consequence, $\mathbf{y}^{t}$ is successfully calibrated by reordering.

\begin{algorithm}[!t]
	\caption{Algorithm of calibration by reordering}
	\label{ag:cr}
	\begin{algorithmic}[1]
		\Require Given a $\mathbf{y}^{t}$ and its corresponding $\mathbf{y}^{h}$
		\Ensure The reordered label $\mathbf{y}^{t_{\zeta}}$
		\State Sort $\mathbf{y}^{t}$ to get $\mathbf{y}^{t_{s}}=\{y^{t}_{1st},y^{t}_{2nd},\dots\,y^{t}_{Cth}\}$
		\State $y^{t}_{tmp}\gets y^{t}_{1st}$
		\For{$ith=1st,2nd,\dots$, until $ith=c$}
		\State $y^{t}_{ith}\gets y^{t}_{(i+1)th}$
		\EndFor
		\State $y^{t}_{c}\gets y^{t}_{tmp}$
	\end{algorithmic}
\end{algorithm}

\subsubsection{Analysis of CT and CR}
\label{sssc:actcr}
We theoretically analyzed the difference between CT and CR in terms of calibration. To be specific, for one $\mathbf{y}^{t}$ that is not located in the correct area in the label space, KDC will be utilized to calibrate $\mathbf{y}^{t}$ as $\mathbf{y}^{t_{\zeta}}$ either by CT or CR, where $\mathbf{y}^{t_{\zeta}}$ is located in the correct area. It can be concluded that there must exist a mapping of any label from the incorrect area to the correct area in the label space. Thus, we wish to know the mapping relationship by calculating the distance between $\mathbf{y}^{h}$ and $\mathbf{y}^{t}$ after KDC.

Fig. \ref{fg:mctcr} shows an example in $3$-D space, where the color map of the distance between $\mathbf{y}^{h}$ and $\mathbf{y}^{t}$ after KDC in the label space is illustrated. The distance calculating function is defined as $\mathcal{D}(\mathbf{y}^{h},\mathbf{y}^{t})=\Vert\mathbf{y}^{h}-\mathbf{y}^{t}\Vert_{2}$. As $\mathbf{y}^{t}$ in the red area (Correct area) do not require calibration, their distance is calculated by $\mathcal{D}(\mathbf{y}^{h},\mathbf{y}^{t})$ directly. On the contrary, $\mathbf{y}^{t}$ out the red area (Incorrect area) are calibrated as $\mathbf{y}^{t_{\zeta}}$. Therefore, their distance is calculated by $\mathcal{D}(\mathbf{y}^{h},\mathbf{y}^{t_{\zeta}})$. As shown in Fig. \ref{fg:mctcr}(a), the $\mathbf{y}^{t}$ being close to the edge of the red area are mapped close to the vertex, i.e. the hard label $\mathbf{y}^{h}$, while the $\mathbf{y}^{t}$ being far away from the red area are mapped to the edge of the red area. Yet in Fig. \ref{fg:mctcr}(b), the $\mathbf{y}^{t}$ being close to the edge of the red area are mapped also close to the edge, while the $\mathbf{y}^{t}$ being far away from the red area are mapped to close to $\mathbf{y}^{h}$. As a consequence, CT keeps the relative position of incorrect $\mathbf{y}^{t}$ unchanged, which means it preserves the spatial information of $\mathbf{y}^{t}$. In addition, CR keeps the shape of the distribution of incorrect $\mathbf{y}^{t}$ unchanged, which means it preserves the distributional information of $\mathbf{y}^{t}$.

\begin{figure}[!t]
	\centering
	\subfigure[Calibration by traslating]{
		\includegraphics[width=0.45\columnwidth]{Figure/kdctm.png}
	}
	\hfil
	\subfigure[Calibration by reordering]{
		\includegraphics[width=0.45\columnwidth]{Figure/kdcrm.png}
	}
	\caption{A $3$-D example illustrating the color map of the distance between $\mathbf{y}^{h}$ and $\mathbf{y}^{t}$ after KDC in the label space}
	\label{fg:mctcr}
\end{figure}

To this end, we are in the position to define the loss function of KDC. Unlike the loss function of KD, we employ KL-divergence only without temperature $\tau$ and the Cross-Entropy part, as shown in Eq. \ref{eq:kdc}.
	
\begin{equation}
\label{eq:kdc}
	\mathcal{L}_{KDC}(\mathbf{y}^{t_{\zeta}},\mathbf{\hat{y}})=\mathcal{L}_{KL}(\mathbf{y}^{t_{\zeta}},\mathbf{\hat{y}})=\sum_{i=1}^{C}y_{i}^{t_{\zeta}}\log\frac{y_{i}^{t_{\zeta}}}{\hat{y}_{i}}
\end{equation}
where $\mathbf{y}^{t_{\zeta}}$ is the calibrated label generated by the teacher model, while $\mathbf{\hat{y}}$ is the label generated by the student model. Compared to $\mathcal{L}_{KD}(\mathbf{y}^{h},\mathbf{\hat{y}},\mathbf{y}^{t_{\tau}},\mathbf{\hat{y}}^{\tau})$, $\mathcal{L}_{KDC}(\mathbf{y}^{t_{\zeta}},\mathbf{\hat{y}})$ does not contains any hyperparameter and computes only KL divergence, which reduces much computational time and the complexity of hyperparameter tuning.
	
Thus, the total process of KDC can be concluded as $3$ steps: 1) Train a teacher model with a heavier and more complex architecture by true labels (Hard labels). Generate the predicted labels by the teacher model; 2) Calibrate the incorrectly predicted labels; 3) Train the student model with a relatively small and simple architecture by calibrated labels (Soft labels). The algorithm of KDC is given in Algorithm \ref{ag:kdc}.
	
\begin{algorithm}[!t]
\caption{Algorithm of KDCTime}
\label{ag:kdc}
\begin{algorithmic}[1]
	\Require The training data $\mathbf{X}$ and its corresponding labels $\mathbf{Y}$.
	\Ensure The trained student model $\mathcal{F}_{S}^{*}$
	\State Initialize a teacher model $\mathcal{F}_{T}$
	\State Train $\mathcal{F}_{T}$ by $\mathbf{X}$ and $\mathbf{Y}$ to get $\mathcal{F}_{T}^{*}$
	\State Generate $\mathbf{Y}^{t}$ by $\mathcal{F}_{T}^{*}$
	\For{each $\mathbf{y}^{t}_{i}$}
		\If{$\mathbf{y}^{t}_{i}$ and $\mathbf{y}^{h}_{i}$ belong to distinct class}
			\State Calibrate $\mathbf{y}^{t}_{i}$ to get $\mathbf{y}^{t_{\zeta}}_{i}$\Comment{Eq. \ref{eq:ctlf} or Algorithm \ref{ag:cr}}
		\EndIf
	\EndFor
	\State Initialize the student model $\mathcal{F}_{S}$
	\State Train $\mathcal{F}_{S}$ by $\mathbf{X}$ and $\mathbf{Y}^{t_{\zeta}}$ to get $\mathcal{F}_{S}^{*}$
\end{algorithmic}
\end{algorithm}
	
\section{Experiments}
\label{sc:exp}
We conduct the experiments on UCR datasets, in which there are $128$ datasets. Yet, there are $15$ problematic datasets with $NaN$ (Not a Number) values, since missing data or various time-series lengths. They are listed in Table \ref{tb:pd}. Therefore, the remaining $113$ datasets are selected for experiments.

\begin{table}[!t]
	\centering
	\caption{Problematic Datasets}
	\label{tb:pd}
	\begin{tabular}{ccc}
		\hline
		 & \textbf{Dataset Name} & \\
		\hline
		AllGestureWiimoteX & DodgerLoopDay & GestureMidAirD1 \\
		AllGestureWiimoteY & DodgerLoopGame & GestureMidAirD2 \\
		AllGestureWiimoteZ & DodgerLoopWeekend & GestureMidAirD3 \\
		MelbournePedestrian & PickupGestureWiimoteZ & GesturePebbleZ1 \\
		ShakeGestureWiimoteZ & PLAID & GesturePebbleZ2 \\
		\hline
	\end{tabular}
\end{table}

In the experiments, ROCKET, Softmax Cross-Entropy for InceptionTime (ITime), label smoothing for InceptionTime (LSTime), Knowledge Distillation for InceptionTime (KDTime), and KD with calibration for InceptionTime (KDCTime) are compared, where KDCTime includes two calibrating methods, i.e. KDC by translating (KDCT) and KDC by reordering (KDCR) respectively. Except for ROCKET, all the aforementioned methods can be concluded as ITime-based approaches. At first, in order to find the best hyperparameter for LSTime, KDTime, KDCT, and KDCR, we conducted the experiments of hyperparameter study for those approaches. Note ITime does not have any hyperparameter to be tuned. After that, we tested the accuracy, training time, and test time of the aforementioned approaches.

Similar to \cite{Fawaz:2020}, critical difference diagrams are drawn in the paper for better illustrating the results of different approaches, as the results of $113$ datasets are hard to be depicted clearly. Critical difference diagram is a diagram drawn by the following steps: 1) Execute the Friedman test \cite{Friedman:1940} for rejecting the null hypothesis. 2) Perform the pairwise post-hoc analysis \cite{Benavoli:2016} by a Wilcoxon signed-rank test with Holm's alpha ($5\%$) correction \cite{Garcia:2008}. 3) Visualize the statistical result by \cite{Demsar:2006}, where a thick horizontal line represents that the approaches are not significantly different with respect to results.

Our experiments are conducted on a computer equipped with an Intel Core i$9$-$11900$ CPU at $2.50$GHz, $32$GB memory, and an NVIDIA GeForce RTX $3090$ GPU. The operating system is Windows 10. Additionally, the development environment is Anaconda $4.10.3$ with Python $3.8.8$ and Pytorch $1.9.0$.

\subsection{Hyperparameter Study for ITime-based Approaches}
\label{ssc:hsa}
In this section, we first searched $3$ hyperparameters, i.e. batch size, epoch, and learning rate, on ITime. Thus, those hyperparameters of LSTime, KDTime, KDCT, and KDCR can be determined also, since all these methods are based on InceptionTime. After that, $\varepsilon$ in LSTime, $\varepsilon$ and $\tau$ in KDTime, and $\varepsilon$ in KDCT and KDCR were searched separately.

\subsubsection{Batch Size}
\label{sssc:bs}
First, the batch size was set to $64$ without searching. The reason is that batch size is theoretically larger the better. An extreme case is the full-batch. However, in deep learning tasks, that always leads to a consequence where graphics memory on GPU is infeasible to load the large dataset. Additionally, The batch size and the epoch are depending on each other, i.e. the more batch size represents the more epochs for converging, which means a longer training time. Thus, the batch size is always empirically set to either $16$, $32$, $64$, $128$, or $256$. Hence, $64$ was adopted in the paper.

\subsubsection{Epoch}
\label{sssc:epc}
With a fixed batch size $64$, we compared the accuracy of $5$ different epochs, which were $64$, $128$, $256$, $512$, and $1024$ respectively. The critical difference diagram is given in Fig. \ref{fg:cdep}, where $1024$ epochs own the best accuracy. Yet, $1024$ epochs are of no critical difference with $512$ epochs. Since the training time of DNN-based approaches is slow, $512$ is selected as the epoch in order to save the training time. In addition, we also employed the early stop strategy with patience equals to $80$ epochs for reducing the training time and alleviating overfitting.

\begin{figure}[!t]
	\centering
	\includegraphics[width=1.0\columnwidth]{Figure/epoch.png}
	\caption{Critical difference diagram for different epochs}
	\label{fg:cdep}
\end{figure}

\subsubsection{Learning Rate}
\label{sssc:lr}
The accuracy of $5$ distinct learning rates were tested in total, which were $0.1$, $0.01$, $0.001$, $0.0001$, and $0.00001$. Moreover, learning rate decay was employed in order to stabilize the training process. We leveraged fixed-step decay, also called piecewise constant decay, as the decay strategy, where the step size is set as $35$ and gamma is set as $0.5$. In other words, the learning rate will multiply by $0.5$ for every $35$ epochs. Note there are $256$ epochs, which ensures $7$ times of decay in total. The critical difference diagram is shown in Fig. \ref{fg:cdlr}, where the learning rate being equal to $0.01$ has the highest accuracy. Thus, $0.01$ is employed as the learning rate in the paper.

\begin{figure}[!t]
	\centering
	\includegraphics[width=1.0\columnwidth]{Figure/lr.png}
	\caption{Critical difference diagram for different learning rates}
	\label{fg:cdlr}
\end{figure}

%\subsubsection{Optimizer}
%\label{sssc:opt}
%\colorbox{yellow}{$XXX$}

\subsubsection{$\varepsilon$ in LSTime}
\label{sssc:lst}
The smoothing coefficient $\varepsilon$ (Eq. \eqref{eq:scels}) represents the smoothed level of labels. We tested the accuracy of $5$ different $\varepsilon$ in LSTime, which are $0.1$, $0.3$, $0.5$, $0.7$, and $0.9$. After all, the critical difference diagram is given in Fig. \ref{fg:lst}, where $0.5$ gets the best accuracy. This claims that $\varepsilon$ should not be too small or too big, since a small $\varepsilon$ gives the label little additional information, while a big $\varepsilon$ causes too much information loss from its original class. In addition, the accuracy of big $\varepsilon$ is less than that of small $\varepsilon$, which means information from its original class is important, and it is not a good idea to completely abandon that information.

\begin{figure}[!t]
	\centering
	\includegraphics[width=1.0\columnwidth]{Figure/LS_e.png}
	\caption{Critical difference diagram for different $\varepsilon$ in LSTime}
	\label{fg:lst}
\end{figure}

\subsubsection{$\varepsilon$ and $\tau$ in KDTime}
\label{sssc:kdt}
KDTime contains $2$ hyperparameters, which are $\varepsilon$ and $\tau$ (Eq. \eqref{eq:kdsce}), where $\varepsilon$ controls the weight of losses between the hard label and the smooth label respectively, and $\tau$ fine-tunes the smoothed level of smooth labels (Predicted labels by teacher model). Thus, we tested the accuracy of $5$ distinct $\varepsilon$ in KDTime, which are $0.1$, $0.3$, $0.5$, $0.7$, and $0.9$, while we also compared the accuracy of $5$ different $\tau$, which are $2$, $4$, $8$, $16$, and $32$. The critical diagrams of $\varepsilon$ and $\tau$ are shown in Fig. \ref{fg:kde} and Fig. \ref{fg:kdt}. Fig. \ref{fg:kde} shows that $\varepsilon=0.5$ has the best accuracy. The $\varepsilon$ close to $0$ or $1$ will reduce the accuracy. In addition, Fig. \ref{fg:kdt} shows that $\tau=8$ is the best. Similarly, the accuracy of KDTime will decrease if $\tau$ is too big or too small.

\begin{figure}[!t]
	\centering
	\includegraphics[width=1.0\columnwidth]{Figure/KD_e.png}
	\caption{Critical difference diagram for different $\varepsilon$ in KDTime}
	\label{fg:kde}
\end{figure}

\begin{figure}[!t]
	\centering
	\includegraphics[width=1.0\columnwidth]{Figure/KD_t.png}
	\caption{Critical difference diagram for different $\tau$ in KDTime}
	\label{fg:kdt}
\end{figure}

To this end, we are able to conclude that all the ITime-based methods select $64$ as the batch size, $512$ as the epoch, and $0.01$ as the learning rate. In addition, $\varepsilon$ in LSTime is set to $0.5$, $\varepsilon$ and $\tau$ in KDTime is set to $0.5$ and $8$ respectively. Finally, Adam is adopted as the optimizing algorithm in order to update the model.

\subsection{Accuracy of Different Approaches}
\label{ssc:aa}
In this section, we compared the accuracy of ROCKET, ITime, LSTime, KDTime, KDCTime by Translating (KDCT), and KDCTime by Reordering (KDCR). The accuracy of all approaches is averaged from $10$ times running. In the meantime, the standard deviation of these $10$ accuracies is also calculated. The result is listed in Table \ref{tb:aa} (Appendix \ref{ap:aa}). The number before the $\pm$ sign represents accuracy, while the number after the $\pm$ sign means standard deviation. Besides, the bold number represents the best accuracy or standard deviation among all approaches.

As shown in Table \ref{tb:aa}, KDCR gets the highest accuracy on $54$ datasets, which claims that its accuracy is competitive and promising. Yet, its converging process is not as stable as ROCKET, since ROCKET has the lowest standard deviation on the majority of datasets. Besides, the critical difference diagram is also given for better illustrating the result of Table \ref{tb:aa}, which is shown in Fig. \ref{fg:acc}. Note that Fig. \ref{fg:acc} also demonstrates the accuracy of the teacher model used in KDTime, KDCT, and KDCR, which is an InceptionTime model with $6$ Inception modules. It is also trained $10$ times and the best one is selected as the teacher. That is the reason why its accuracy is the best one. In addition, KDCR and KDCT are of no significant difference with KDTime. The reason is that the accuracy of $75$ datasets are more than $0.8$ for the teacher model, which claims the majority of datasets meets the bottleneck of improving accuracy. In other words, only a small number of samples can be calibrated by KDCTime. Thus, We claim that the results of those $85$ datasets dominate the other $28$ datasets in the critical difference diagram. By ignoring that part of the results, the accuracy of KDCR and KDCT is more promising, as shown in Fig. \ref{fg:acc3}.

In summary, KDCR owns the best accuracy, which is better than KDCT. The reason is that KDCR keeps the information from marginal labels. In Fig. \ref{fg:spc}, marginal labels represent the labels located in the middle area of the label space. In Fig. \ref{fg:mctcr}(b), we know the labels located in the middle area have a long distance to $\mathbf{y}^{h}$, where they do not deterministically belong to any class, meaning that they contain abundant information from other classes. Yet, KDCT calibrates the marginal labels close to $\mathbf{y}^{h}$, which losses that information. In addition, KDCR calibrates the labels close to other classes as close to the correct one, as shown in Fig. \ref{fg:mctcr}(b), which eliminates the misguiding from deterministic incorrect labels. Nevertheless, KDCT keeps the information of those labels from incorrect classes.

\begin{figure}[!t]
	\centering
	\includegraphics[width=1.0\columnwidth]{Figure/acc.png}
	\caption{Critical difference diagram illustrating the accuracy of different approaches}
	\label{fg:acc}
\end{figure}

\begin{figure}[!t]
	\centering
	\includegraphics[width=1.0\columnwidth]{Figure/acc3.png}
	\caption{Critical difference diagram illustrating the accuracy of three approaches on datasets with accuracy less than $0.8$ for KDTime}
	\label{fg:acc3}
\end{figure}

\subsection{Training and Test Time of Different Approaches}
\label{ssc:ta}
In this section, we compared the training and test time of ROCKET, ITime, LSTime, KDTime, and KDCTIme, ignoring KDCT or KDCR, as their training and inference time are of no significant difference. Instead of showing all the results in a table, diagrams of comparison were selected for better illustrating the results.

Fig. \ref{fg:traint} demonstrates the training time of KDCTime compared with ROCKET, ITime, LSTime, and KDTime. In Fig. \ref{fg:traint}(a), it shows that KDCTime is much slower. In detail, the training time of KDCTime on $77$ datasets is below $1$ order of magnitude slower than ROCKET, while that on $36$ datasets is more than $1$ order of magnitude slower. That is because all ITime-based approaches, including KDCTime, employ Gradient Descent as the optimizing algorithm, which requires the inference on many training samples for each update and a large number of updates in total, e.g. $64$ samples for each update, many times of updating in $1$ epoch, and $512$ epochs in total. On the contrary, ROCKET utilizes the ridge classifier and solves the ridge regression problem directly. However, the training time of KDCTime is still acceptable, which requires around $1$ hour to train $113$ UCR datasets and $10$ hours for $10$ times running in total. Besides, as shown in Fig. \ref{fg:traint}(b) and (c), the training time of ITime and LSTime is similar to KDCTime. The reason is that their model is the same, which is InceptionTime with $3$ Inception modules. Still, KDCTime needs an extra teacher model in order to guild the training of its student model, which leads to a consequence that KDCTime requires extra training time for the teacher model. Note that the training of the teacher model is required for only one time. Once the teacher model is obtained, that model is available for multiple times of training for the student model. Fig. \ref{fg:traint}(d) shows that the training time of KDCTime is of no significant difference with KDTime, as their models are the same, they both require the teacher model, and Gradient Descent is selected as their optimizing algorithms. In conclusion, the difference of training time mainly appears in $3$ categories of methods, where the first one is ROCKET, the second one is ITime-based methods without KD, and the last one is ITime-based approaches with KD.

\begin{figure}[!t]
	\centering
	\subfigure[KDCTime and ROCKET]{
		\includegraphics[width=0.45\columnwidth]{Figure/traint_kdcr_rocket.png}
	}
	\hfil
	\subfigure[KDCTime and ITime]{
		\includegraphics[width=0.45\columnwidth]{Figure/traint_kdcr_itime.png}
	}
	\subfigure[KDCTime and LSTime]{
		\includegraphics[width=0.45\columnwidth]{Figure/traint_kdcr_lstime.png}
	}
	\hfil
	\subfigure[KDCTime and KDTime]{
		\includegraphics[width=0.45\columnwidth]{Figure/traint_kdcr_kdtime.png}
	}
	\caption{The training time of KDCTime compared with ROCKET, ITime, LSTime, and KDTime}
	\label{fg:traint}
\end{figure}

Fig. \ref{fg:testt} demonstrates the test time of KDCTime compared with ROCKET, ITime, LSTime, and KDTime. In Fig. \ref{fg:testt}(a), it shows that KDCTime is much faster than ROCKET. To be specific, the test time of KDCTime on $42$ datasets is $1$ order of magnitude faster than ROCKET, that on $61$ datasets is $2$ orders of magnitude faster, and that on $10$ datasets is $3$ orders of magnitude faster. That is because in the stage of test, without the computational time of Gradient Descent, KDCTime only requires the time of inference on test samples, which is the same as ROCKET. In this scenario, the computational time of $10000$ random convolutional kernels in ROCKET is much slower than the $3$ Inception modules in  KDCTime. In addition, as shown in Fig. \ref{fg:testt}(b), (c), and (d), the test time of those approaches are of no difference with KDCTime, since their inference models are all InceptionTime with $3$ Inception modules. As a consequence, the difference of test time can be categorized as $2$ groups, which are ROCKET and ITime-based approaches, regardless of ITime-based approaches with or without KD.

\begin{figure}[!t]
	\centering
	\subfigure[KDCTime and ROCKET]{
		\includegraphics[width=0.45\columnwidth]{Figure/testt_kdcr_rocket.png}
	}
	\hfil
	\subfigure[KDCTime and ITime]{
		\includegraphics[width=0.45\columnwidth]{Figure/testt_kdcr_itime.png}
	}
	\subfigure[KDCTime and LSTime]{
		\includegraphics[width=0.45\columnwidth]{Figure/testt_kdcr_lstime.png}
	}
	\hfil
	\subfigure[KDCTime and KDTime]{
		\includegraphics[width=0.45\columnwidth]{Figure/testt_kdcr_kdtime.png}
	}
	\caption{The test time of KDCTime compared with ROCKET, ITime, LSTime, and KDTime}
	\label{fg:testt}
\end{figure}

\section{Conclusion}
\label{sc:conc}
In this paper, we discovered the DNN-based TSC approaches are easy to be overfitting on the UCR datasets, which is caused by the few-shot problem in the UCR archive. Thus, in order to alleviate overfitting, Label Smoothing for InceptionTime (LSTime) was first proposed by utilizing soft labels. Next, instead of manually adjusting soft labels, Knowledge Distillation for InceptionTime (KDTime) was proposed in order to automatically generate soft labels. At last, in order to rectify the incorrect predicted soft labels from the teacher model, KD with calibration (KDC) was proposed, where it has two optional strategies, namely KDC by Translating (KDCT) and KDC by Reordering (KDCR).

The experimental results show that the accuracy of KDCT and KDCR is promising, while KDCR gets the highest one. In addition, including KDCT and KDCR, all InceptionTime-based (ITime-based) approaches are $2$ orders of magnitude faster than ROCKET on test time, since the ITime model is the majority factor for the inference time. The training time of ITime-based approaches is slower than ROCKET, yet it is in an acceptable range and worthwhile in order to obtain a promising accuracy and fast inference time. At last, KDCT and KDCR do not introduce any additional hyperparameter compared to ITime.

In the future, instead of just concentrating on the loss functions and labels, we will try various models, in order to propose a brand new model which owns a high generalization capability.




% Reminder: the "draftcls" or "draftclsnofoot", not "draft", class
% option should be used if it is desired that the figures are to be
% displayed while in draft mode.
%
%\begin{figure}[!t]
%\centering
%\includegraphics[width=2.5in]{myfigure}
% where an .eps filename suffix will be assumed under latex, 
% and a .pdf suffix will be assumed for pdflatex; or what has been declared
% via \DeclareGraphicsExtensions.
%\caption{Simulation results for the network.}
%\label{fig_sim}
%\end{figure}

% Note that the IEEE typically puts floats only at the top, even when this
% results in a large percentage of a column being occupied by floats.


% An example of a double column floating figure using two subfigures.
% (The subfig.sty package must be loaded for this to work.)
% The subfigure \label commands are set within each subfloat command,
% and the \label for the overall figure must come after \caption.
% \hfil is used as a separator to get equal spacing.
% Watch out that the combined width of all the subfigures on a 
% line do not exceed the text width or a line break will occur.
%
%\begin{figure*}[!t]
%\centering
%\subfloat[Case I]{\includegraphics[width=2.5in]{box}%
%\label{fig_first_case}}
%\hfil
%\subfloat[Case II]{\includegraphics[width=2.5in]{box}%
%\label{fig_second_case}}
%\caption{Simulation results for the network.}
%\label{fig_sim}
%\end{figure*}
%
% Note that often IEEE papers with subfigures do not employ subfigure
% captions (using the optional argument to \subfloat[]), but instead will
% reference/describe all of them (a), (b), etc., within the main caption.
% Be aware that for subfig.sty to generate the (a), (b), etc., subfigure
% labels, the optional argument to \subfloat must be present. If a
% subcaption is not desired, just leave its contents blank,
% e.g., \subfloat[].


% An example of a floating table. Note that, for IEEE style tables, the
% \caption command should come BEFORE the table and, given that table
% captions serve much like titles, are usually capitalized except for words
% such as a, an, and, as, at, but, by, for, in, nor, of, on, or, the, to
% and up, which are usually not capitalized unless they are the first or
% last word of the caption. Table text will default to \footnotesize as
% the IEEE normally uses this smaller font for tables.
% The \label must come after \caption as always.
%
%\begin{table}[!t]
%% increase table row spacing, adjust to taste
%\renewcommand{\arraystretch}{1.3}
% if using array.sty, it might be a good idea to tweak the value of
% \extrarowheight as needed to properly center the text within the cells
%\caption{An Example of a Table}
%\label{table_example}
%\centering
%% Some packages, such as MDW tools, offer better commands for making tables
%% than the plain LaTeX2e tabular which is used here.
%\begin{tabular}{|c||c|}
%\hline
%One & Two\\
%\hline
%Three & Four\\
%\hline
%\end{tabular}
%\end{table}


% Note that the IEEE does not put floats in the very first column
% - or typically anywhere on the first page for that matter. Also,
% in-text middle ("here") positioning is typically not used, but it
% is allowed and encouraged for Computer Society conferences (but
% not Computer Society journals). Most IEEE journals/conferences use
% top floats exclusively. 
% Note that, LaTeX2e, unlike IEEE journals/conferences, places
% footnotes above bottom floats. This can be corrected via the
% \fnbelowfloat command of the stfloats package.


% use section* for acknowledgment
%\section*{Acknowledgment}
%The authors would like to thank...

\appendices
\section{The procedure to calculate $\delta$}
\label{ap:pcd}
Given a hard label $\mathbf{y}^{h}$ and a soft label $\mathbf{y}^{t}$, where they both satisfy $\sum_{i=1}^{C}y_{i}=1$ and $\forall i, y_{i}\geq0$. By treating $\mathbf{y}^{h}$ and $\mathbf{y}^{t}$ as vectors, we want to find the minimum distance between them when $\mathop{\arg\max}_{i}\{y^{t}_{i}\}\neq\mathop{\arg\max}_{i}\{y^{h}_{i}\}$. Let $c=\mathop{\arg\max}_{i}\{y^{h}_{i}\}$, so that $y^{h}_{c}=1$ and $c\neq\mathop{\arg\max}_{i}\{y^{t}_{i}\}$. Let $m=\mathop{\arg\max}_{i}\{y^{t}_{i}\}$, so that $\forall i,y^{t}_{i}\leq y^{t}_{m}$. Therefore, we have the following optimization objective:

\begin{equation*}
	\label{eq:doo}
	\begin{aligned}
		\min&\Vert\mathbf{y}^{h}-\mathbf{y}^{t}\Vert_{2}\\
		\mathrm{s.t.}&\sum_{i=1}^{C}y^{t}_{i}=1\\
		&y^{t}_{i}\geq0, i=1,2,\dots,C\\
		&y^{t}_{i}\leq y^{t}_{m}, i=1,2,\dots,C
	\end{aligned}
\end{equation*}
where we know $y^{h}_{c}=1$ and $\forall i\neq c,y^{h}_{i}=0$. Thus, $\min\Vert\mathbf{y}^{h}-\mathbf{y}^{t}\Vert_{2}=\min\{(1-y^{t}_{c})^{2}+\sum_{i\neq c}(y^{t}_{i})^{2}\}$. Since $\sum_{i=1}^{C}y^{t}_{i}=1$, we let $y^{t}_{c}=1-\sum_{i\neq c}y^{t}_{i}$. Thus, $\min\{(1-y^{t}_{c})^{2}+\sum_{i\neq c}(y^{t}_{i})^{2}\}=\min\{(\sum_{i\neq c}y^{t}_{i})^{2}+\sum_{i\neq c}(y^{t}_{i})^{2}\}$. In this way, our optimization objective can be rewritten as follows:

\begin{equation*}
	\label{eq:dood}
	\begin{aligned}
		\min&\{(\sum_{i\neq c}y^{t}_{i})^{2}+\sum_{i\neq c}(y^{t}_{i})^{2}\}\\
		\mathrm{s.t.}&-y^{t}_{i}\leq0,i=1,2,\dots,C\\
		&y^{t}_{i}-y^{t}_{m}\leq0,i=1,2,\dots,C
	\end{aligned}
\end{equation*}

This is an optimization problem with inequality constraints. Therefore, we can define its Lagrangian function as:

\begin{equation*}
\label{eq:lag}
\begin{aligned}
	&\mathcal{L}(\mathbf{y}^{t},\boldsymbol{\lambda},\boldsymbol{\mu})\\
	&=(\sum_{i\neq c}y^{t}_{i})^{2}+\sum_{i\neq c}(y^{t}_{i})^{2}-\sum_{i }\lambda_{i}y^{t}_{i}+\sum_{i }\mu_{i}(y^{t}_{i}-y^{t}_{m})
\end{aligned}
\end{equation*}
where $\boldsymbol{\lambda}\in \mathbb{R}^{C}$ and $\boldsymbol{\mu}\in \mathbb{R}^{C}$ are two sets of Lagrangian multipliers. By adopting Karush-Kuhn-Tucker (KKT) Conditions, we have:

\begin{equation*}
\label{eq:kkt}
	\begin{cases}
		-\lambda_{c}+\mu_{c}=0,i=c\\
		2\sum_{j\neq c}y^{t}_{j}+2y^{t}_{m}-\lambda_{m}-\sum_{j\neq m}\mu_{j}=0,i=m\\
		2\sum_{j\neq c}y^{t}_{j}+2y^{t}_{i}-\lambda_{i}+\mu_{i}=0,i\neq c~and~i\neq m\\
		-\lambda_{i}y^{t}_{i}=0,i=0,1,\dots,C\\
		\mu_{i}(y^{t}_{i}-y^{t}_{m})=0,i=0,1,\dots,C\\
		\lambda_{i}\geq0,i=0,1,\dots,C\\
		\mu_{i}\geq0,i=0,1,\dots,C
	\end{cases}
\end{equation*}

At last, after solving this system of equations, we know $\mathcal{L}(\mathbf{y}^{t},\boldsymbol{\lambda},\boldsymbol{\mu})$ obtains the minimum value when $y^{t}_{c}=1/2$, $y^{t}_{m}=1/2$, and all other $y^{t}_{i}=0$. As a result, $\delta$ can be calculated as follows:

\begin{equation*}
\label{eq:delta}
	\delta=\min\Vert\mathbf{y}^{h}-\mathbf{y}^{t}\Vert_{2}=\sqrt{(1-\frac{1}{2})^{2}+(-\frac{1}{2})^2}=\frac{1}{\sqrt{2}}
\end{equation*}

\section{The accuracy of different approaches}
\label{ap:aa}
The accuracy of different approaches on UCR datasets is given in Table \ref{tb:aa} on the last page.

\begin{table*}[!t]\tiny
	\centering
	\caption{The accuracy of different approaches}
	\label{tb:aa}
	\begin{tabular}{ccccccc}
		\hline
		\textbf{Datasets} & \textbf{ROCKET} & \textbf{ITime} & \textbf{LSTime} & \textbf{KDTime} & \textbf{KDCT} & \textbf{KDCR}\\
		\hline
		ACSF1 & $0.882\pm0.01$ & $0.896\pm0.007$ & $0.903\pm\mathbf{0.004}$ & $0.911\pm0.021$ & $\mathbf{0.914}\pm0.006$ & $0.913\pm0.009$ \\
		Adiac & $0.8\pm\mathbf{0.005}$ & $0.822\pm0.02$ & $0.791\pm0.007$ & $0.842\pm0.007$ & $\mathbf{0.847}\pm0.009$ & $0.826\pm0.014$ \\
		ArrowHead & $0.807\pm\mathbf{0.009}$ & $0.84\pm0.018$ & $0.854\pm0.012$ & $0.857\pm0.016$ & $0.859\pm0.025$ & $\mathbf{0.873}\pm0.019$ \\
		Beef & $\mathbf{0.8}\pm\mathbf{0.0}$ & $0.767\pm\mathbf{0.0}$ & $0.767\pm\mathbf{0.0}$ & $0.793\pm0.057$ & $0.793\pm0.025$ & $\mathbf{0.8}\pm0.047$ \\
		BeetleFly & $0.9\pm\mathbf{0.0}$ & $0.74\pm0.02$ & $0.8\pm0.041$ & $0.87\pm0.024$ & $\mathbf{0.94}\pm0.02$ & $0.83\pm0.06$ \\
		BirdChicken & $0.89\pm\mathbf{0.02}$ & $0.92\pm0.024$ & $0.917\pm0.085$ & $\mathbf{0.95}\pm0.055$ & $0.92\pm0.024$ & $\mathbf{0.95}\pm0.063$ \\
		BME & $\mathbf{1.0}\pm\mathbf{0.0}$ & $0.995\pm\mathbf{0.0}$ & $0.995\pm0.004$ & $0.999\pm0.002$ & $0.992\pm0.003$ & $\mathbf{1.0}\pm0.002$ \\
		Car & $0.917\pm\mathbf{0.0}$ & $0.91\pm0.017$ & $0.867\pm0.014$ & $\mathbf{0.937}\pm0.012$ & $0.91\pm0.023$ & $0.917\pm0.008$ \\
		CBF & $\mathbf{1.0}\pm\mathbf{0.0}$ & $0.998\pm0.001$ & $0.998\pm0.001$ & $0.999\pm\mathbf{0.0}$ & $0.999\pm\mathbf{0.0}$ & $\mathbf{1.0}\pm\mathbf{0.0}$ \\
		Chinatown & $0.977\pm\mathbf{0.0}$ & $0.984\pm0.005$ & $\mathbf{0.992}\pm\mathbf{0.0}$ & $\mathbf{0.992}\pm0.001$ & $0.991\pm0.002$ & $\mathbf{0.992}\pm\mathbf{0.0}$ \\
		ChlorineConcentration & $0.81\pm0.006$ & $0.839\pm0.007$ & $0.83\pm0.016$ & $0.864\pm\mathbf{0.004}$ & $0.848\pm0.01$ & $\mathbf{0.873}\pm0.005$ \\
		CinCECGTorso & $0.826\pm\mathbf{0.005}$ & $0.863\pm0.009$ & $0.846\pm0.007$ & $0.868\pm0.012$ & $0.871\pm0.009$ & $\mathbf{0.872}\pm0.008$ \\
		Coffee & $\mathbf{1.0}\pm\mathbf{0.0}$ & $\mathbf{1.0}\pm\mathbf{0.0}$ & $\mathbf{1.0}\pm\mathbf{0.0}$ & $\mathbf{1.0}\pm\mathbf{0.0}$ & $\mathbf{1.0}\pm\mathbf{0.0}$ & $\mathbf{1.0}\pm\mathbf{0.0}$ \\
		Computers & $0.761\pm0.005$ & $0.813\pm0.014$ & $0.831\pm0.005$ & $0.826\pm\mathbf{0.003}$ & $\mathbf{0.851}\pm0.014$ & $0.818\pm0.01$ \\
		CricketX & $0.829\pm0.004$ & $0.855\pm0.014$ & $0.852\pm0.005$ & $0.86\pm\mathbf{0.002}$ & $\mathbf{0.871}\pm0.009$ & $0.85\pm0.008$ \\
		CricketY & $0.856\pm\mathbf{0.002}$ & $0.849\pm0.017$ & $0.873\pm0.008$ & $0.875\pm0.006$ & $0.879\pm0.008$ & $\mathbf{0.893}\pm0.009$ \\
		CricketZ & $0.858\pm\mathbf{0.003}$ & $0.856\pm0.007$ & $0.858\pm\mathbf{0.003}$ & $\mathbf{0.867}\pm0.006$ & $0.865\pm0.008$ & $0.865\pm0.007$ \\
		Crop & $0.763\pm\mathbf{0.001}$ & $0.77\pm0.002$ & $0.776\pm0.003$ & $0.777\pm0.002$ & $0.776\pm\mathbf{0.001}$ & $\mathbf{0.788}\pm\mathbf{0.001}$ \\
		DiatomSizeReduction & $\mathbf{0.975}\pm\mathbf{0.001}$ & $0.93\pm0.005$ & $0.938\pm0.023$ & $0.965\pm0.013$ & $0.942\pm0.013$ & $0.951\pm0.003$ \\
		DistalPhalanxOutlineAgeGroup & $0.758\pm\mathbf{0.006}$ & $0.795\pm0.011$ & $\mathbf{0.845}\pm0.01$ & $0.844\pm0.012$ & $0.836\pm0.007$ & $0.818\pm0.01$ \\
		DistalPhalanxOutlineCorrect & $0.773\pm\mathbf{0.005}$ & $0.762\pm0.017$ & $0.815\pm0.009$ & $0.816\pm0.011$ & $\mathbf{0.818}\pm0.011$ & $0.808\pm0.007$ \\
		DistalPhalanxTW & $\mathbf{0.717}\pm0.007$ & $0.626\pm0.041$ & $\mathbf{0.717}\pm\mathbf{0.004}$ & $0.635\pm0.018$ & $0.711\pm0.016$ & $0.713\pm0.011$ \\
		Earthquakes & $\mathbf{0.748}\pm\mathbf{0.0}$ & $0.647\pm0.038$ & $0.724\pm0.006$ & $0.736\pm0.013$ & $0.742\pm0.013$ & $0.742\pm0.012$ \\
		ECG200 & $0.906\pm\mathbf{0.005}$ & $0.887\pm0.013$ & $0.905\pm0.009$ & $0.915\pm0.01$ & $0.921\pm\mathbf{0.005}$ & $\mathbf{0.931}\pm0.007$ \\
		ECG5000 & $\mathbf{0.947}\pm\mathbf{0.0}$ & $0.94\pm\mathbf{0.0}$ & $0.946\pm\mathbf{0.0}$ & $0.945\pm0.001$ & $\mathbf{0.947}\pm\mathbf{0.0}$ & $0.944\pm0.001$ \\
		ECGFiveDays & $\mathbf{1.0}\pm\mathbf{0.0}$ & $0.998\pm0.001$ & $0.999\pm0.001$ & $\mathbf{1.0}\pm0.001$ & $0.999\pm0.001$ & $\mathbf{1.0}\pm\mathbf{0.0}$ \\
		ElectricDevices & $0.73\pm\mathbf{0.002}$ & $0.697\pm0.006$ & $0.733\pm0.008$ & $0.741\pm0.004$ & $0.738\pm0.006$ & $\mathbf{0.756}\pm0.004$ \\
		EOGHorizontalSignal & $0.578\pm\mathbf{0.004}$ & $0.661\pm0.005$ & $0.651\pm0.007$ & $0.691\pm0.01$ & $0.693\pm0.005$ & $\mathbf{0.713}\pm0.007$ \\
		EOGVerticalSignal & $0.547\pm0.006$ & $0.496\pm0.014$ & $0.497\pm\mathbf{0.005}$ & $0.541\pm0.017$ & $0.542\pm0.011$ & $\mathbf{0.563}\pm0.009$ \\
		EthanolLevel & $0.588\pm0.008$ & $0.783\pm0.017$ & $0.705\pm0.017$ & $0.762\pm\mathbf{0.005}$ & $\mathbf{0.817}\pm0.01$ & $0.816\pm0.02$ \\
		FaceAll & $0.923\pm\mathbf{0.007}$ & $0.816\pm0.01$ & $0.833\pm0.009$ & $0.897\pm0.028$ & $0.912\pm0.026$ & $\mathbf{0.963}\pm0.022$ \\
		FaceFour & $\mathbf{0.975}\pm0.005$ & $0.956\pm\mathbf{0.0}$ & $0.956\pm\mathbf{0.0}$ & $0.96\pm0.008$ & $0.954\pm0.003$ & $0.956\pm0.008$ \\
		FacesUCR & $0.962\pm\mathbf{0.001}$ & $0.965\pm0.002$ & $0.964\pm0.002$ & $0.968\pm\mathbf{0.001}$ & $0.968\pm0.002$ & $\mathbf{0.973}\pm\mathbf{0.001}$ \\
		FiftyWords & $\mathbf{0.835}\pm\mathbf{0.001}$ & $0.729\pm0.003$ & $0.63\pm0.006$ & $0.677\pm0.022$ & $0.753\pm0.015$ & $0.755\pm0.014$ \\
		Fish & $0.983\pm\mathbf{0.0}$ & $0.984\pm0.003$ & $0.99\pm0.004$ & $\mathbf{0.992}\pm0.004$ & $0.99\pm0.003$ & $0.99\pm0.003$ \\
		FordA & $0.944\pm0.001$ & $0.948\pm0.002$ & $0.958\pm\mathbf{0.0}$ & $\mathbf{0.963}\pm0.001$ & $0.961\pm0.001$ & $0.962\pm0.001$ \\
		FordB & $0.803\pm0.005$ & $0.838\pm0.003$ & $0.856\pm0.003$ & $0.859\pm0.004$ & $0.861\pm\mathbf{0.002}$ & $\mathbf{0.863}\pm0.004$ \\
		FreezerRegularTrain & $0.998\pm\mathbf{0.0}$ & $0.997\pm\mathbf{0.0}$ & $\mathbf{0.999}\pm\mathbf{0.0}$ & $\mathbf{0.999}\pm\mathbf{0.0}$ & $\mathbf{0.999}\pm\mathbf{0.0}$ & $0.998\pm\mathbf{0.0}$ \\
		FreezerSmallTrain & $\mathbf{0.952}\pm\mathbf{0.004}$ & $0.872\pm\mathbf{0.004}$ & $0.874\pm0.008$ & $0.937\pm0.006$ & $0.935\pm0.008$ & $0.944\pm0.007$ \\
		Fungi & $0.987\pm0.003$ & $0.999\pm\mathbf{0.002}$ & $0.995\pm0.004$ & $0.997\pm0.003$ & $0.997\pm0.003$ & $\mathbf{1.0}\pm0.005$ \\
		GunPoint & $\mathbf{1.0}\pm\mathbf{0.0}$ & $\mathbf{1.0}\pm\mathbf{0.0}$ & $0.998\pm0.002$ & $0.999\pm0.002$ & $\mathbf{1.0}\pm\mathbf{0.0}$ & $\mathbf{1.0}\pm0.006$ \\
		GunPointAgeSpan & $0.992\pm0.003$ & $0.985\pm0.007$ & $0.992\pm\mathbf{0.001}$ & $0.992\pm0.005$ & $0.992\pm0.004$ & $\mathbf{0.994}\pm0.009$ \\
		GunPointMaleVersusFemale & $0.999\pm0.002$ & $\mathbf{1.0}\pm\mathbf{0.0}$ & $\mathbf{1.0}\pm\mathbf{0.0}$ & $\mathbf{1.0}\pm\mathbf{0.0}$ & $\mathbf{1.0}\pm\mathbf{0.0}$ & $\mathbf{1.0}\pm0.002$ \\
		GunPointOldVersusYoung & $\mathbf{1.0}\pm\mathbf{0.0}$ & $\mathbf{1.0}\pm\mathbf{0.0}$ & $\mathbf{1.0}\pm\mathbf{0.0}$ & $\mathbf{1.0}\pm\mathbf{0.0}$ & $\mathbf{1.0}\pm\mathbf{0.0}$ & $\mathbf{1.0}\pm\mathbf{0.0}$ \\
		Ham & $0.722\pm0.02$ & $0.767\pm0.02$ & $0.816\pm\mathbf{0.009}$ & $\mathbf{0.821}\pm0.012$ & $0.797\pm0.011$ & $0.818\pm0.029$ \\
		HandOutlines & $0.944\pm\mathbf{0.002}$ & $0.94\pm0.005$ & $0.911\pm0.011$ & $0.945\pm0.009$ & $0.95\pm0.006$ & $\mathbf{0.954}\pm0.01$ \\
		Haptics & $0.523\pm\mathbf{0.0}$ & $0.552\pm0.014$ & $0.576\pm0.012$ & $0.577\pm0.006$ & $\mathbf{0.579}\pm0.013$ & $0.562\pm0.007$ \\
		Herring & $0.691\pm\mathbf{0.012}$ & $0.725\pm0.035$ & $\mathbf{0.75}\pm0.013$ & $0.728\pm0.016$ & $0.707\pm0.023$ & $0.743\pm0.025$ \\
		HouseTwenty & $\mathbf{0.965}\pm\mathbf{0.003}$ & $0.934\pm\mathbf{0.003}$ & $0.948\pm0.004$ & $0.953\pm0.012$ & $0.943\pm0.007$ & $0.944\pm0.016$ \\
		InlineSkate & $0.456\pm\mathbf{0.004}$ & $0.446\pm0.015$ & $0.418\pm0.007$ & $\mathbf{0.474}\pm0.015$ & $0.471\pm0.031$ & $0.458\pm0.008$ \\
		InsectEPGRegularTrain & $\mathbf{1.0}\pm\mathbf{0.0}$ & $\mathbf{1.0}\pm\mathbf{0.0}$ & $\mathbf{1.0}\pm\mathbf{0.0}$ & $\mathbf{1.0}\pm\mathbf{0.0}$ & $\mathbf{1.0}\pm\mathbf{0.0}$ & $\mathbf{1.0}\pm\mathbf{0.0}$ \\
		InsectEPGSmallTrain & $0.998\pm0.002$ & $\mathbf{1.0}\pm\mathbf{0.0}$ & $\mathbf{1.0}\pm\mathbf{0.0}$ & $\mathbf{1.0}\pm\mathbf{0.0}$ & $\mathbf{1.0}\pm\mathbf{0.0}$ & $\mathbf{1.0}\pm\mathbf{0.0}$ \\
		InsectWingbeatSound & $\mathbf{0.652}\pm\mathbf{0.001}$ & $0.62\pm0.009$ & $0.635\pm0.005$ & $0.635\pm0.007$ & $0.631\pm0.007$ & $0.64\pm0.004$ \\
		ItalyPowerDemand & $0.969\pm\mathbf{0.0}$ & $0.972\pm0.002$ & $0.973\pm0.002$ & $0.974\pm0.001$ & $0.974\pm0.001$ & $\mathbf{0.975}\pm0.001$ \\
		LargeKitchenAppliances & $0.888\pm0.003$ & $0.89\pm0.008$ & $0.891\pm\mathbf{0.002}$ & $0.91\pm0.004$ & $0.897\pm0.005$ & $\mathbf{0.914}\pm0.006$ \\
		Lightning2 & $0.77\pm\mathbf{0.0}$ & $0.813\pm0.017$ & $0.831\pm0.008$ & $0.856\pm0.019$ & $0.859\pm0.017$ & $\mathbf{0.869}\pm0.028$ \\
		Lightning7 & $0.841\pm0.011$ & $0.822\pm0.027$ & $0.906\pm0.011$ & $0.905\pm\mathbf{0.009}$ & $0.916\pm0.012$ & $\mathbf{0.938}\pm0.046$ \\
		Mallat & $0.955\pm\mathbf{0.001}$ & $0.95\pm0.006$ & $0.834\pm0.011$ & $0.947\pm0.006$ & $0.945\pm0.007$ & $\mathbf{0.963}\pm0.014$ \\
		Meat & $0.95\pm\mathbf{0.0}$ & $0.913\pm0.019$ & $0.928\pm0.031$ & $0.913\pm0.037$ & $0.917\pm0.028$ & $\mathbf{0.967}\pm0.012$ \\
		MedicalImages & $\mathbf{0.802}\pm\mathbf{0.003}$ & $0.761\pm0.008$ & $0.794\pm0.007$ & $0.786\pm\mathbf{0.003}$ & $0.79\pm0.005$ & $0.779\pm0.004$ \\
		MiddlePhalanxOutlineAgeGroup & $0.603\pm\mathbf{0.003}$ & $0.562\pm0.01$ & $0.708\pm0.004$ & $0.582\pm\mathbf{0.003}$ & $\mathbf{0.71}\pm0.006$ & $0.706\pm0.019$ \\
		MiddlePhalanxOutlineCorrect & $0.839\pm0.007$ & $0.824\pm0.011$ & $0.853\pm0.004$ & $\mathbf{0.858}\pm0.007$ & $0.855\pm\mathbf{0.002}$ & $0.854\pm0.006$ \\
		MiddlePhalanxTW & $0.552\pm0.006$ & $0.498\pm0.026$ & $0.61\pm\mathbf{0.002}$ & $0.617\pm0.006$ & $\mathbf{0.62}\pm0.009$ & $0.598\pm0.03$ \\
		MixedShapesRegularTrain & $0.971\pm\mathbf{0.001}$ & $0.975\pm0.002$ & $\mathbf{0.976}\pm\mathbf{0.001}$ & $\mathbf{0.976}\pm\mathbf{0.001}$ & $0.974\pm\mathbf{0.001}$ & $0.975\pm0.002$ \\
		MixedShapesSmallTrain & $\mathbf{0.937}\pm\mathbf{0.0}$ & $0.932\pm0.003$ & $0.932\pm0.004$ & $\mathbf{0.937}\pm0.001$ & $0.935\pm0.006$ & $0.935\pm0.006$ \\
		MoteStrain & $\mathbf{0.91}\pm\mathbf{0.002}$ & $0.9\pm0.008$ & $0.908\pm0.008$ & $0.896\pm0.005$ & $0.908\pm0.004$ & $0.874\pm0.026$ \\
		NonInvasiveFetalECGThorax1 & $0.954\pm0.002$ & $0.958\pm0.003$ & $0.952\pm0.005$ & $0.961\pm0.003$ & $\mathbf{0.964}\pm0.003$ & $\mathbf{0.964}\pm\mathbf{0.001}$ \\
		NonInvasiveFetalECGThorax2 & $\mathbf{0.97}\pm\mathbf{0.001}$ & $0.959\pm0.004$ & $0.96\pm0.002$ & $0.961\pm0.003$ & $0.961\pm\mathbf{0.001}$ & $0.961\pm0.002$ \\
		OliveOil & $\mathbf{0.907}\pm\mathbf{0.013}$ & $0.753\pm0.016$ & $0.722\pm0.079$ & $0.633\pm0.191$ & $0.64\pm0.196$ & $0.767\pm0.033$ \\
		OSULeaf & $0.939\pm\mathbf{0.004}$ & $0.97\pm0.007$ & $0.97\pm\mathbf{0.004}$ & $\mathbf{0.981}\pm\mathbf{0.004}$ & $0.978\pm0.005$ & $0.972\pm0.006$ \\
		PhalangesOutlinesCorrect & $0.836\pm0.003$ & $0.831\pm0.008$ & $\mathbf{0.855}\pm\mathbf{0.002}$ & $\mathbf{0.855}\pm0.008$ & $0.848\pm0.003$ & $0.851\pm0.003$ \\
		Phoneme & $0.281\pm\mathbf{0.002}$ & $0.339\pm0.007$ & $0.344\pm0.006$ & $\mathbf{0.346}\pm0.004$ & $\mathbf{0.346}\pm0.003$ & $0.331\pm0.003$ \\
		PigAirwayPressure & $\mathbf{0.834}\pm0.028$ & $0.577\pm0.018$ & $0.367\pm0.032$ & $0.418\pm0.034$ & $0.592\pm0.02$ & $0.57\pm\mathbf{0.008}$ \\
		PigArtPressure & $0.962\pm0.004$ & $0.995\pm0.002$ & $0.863\pm0.006$ & $\mathbf{0.996}\pm\mathbf{0.0}$ & $\mathbf{0.996}\pm\mathbf{0.0}$ & $0.992\pm0.014$ \\
		PigCVP & $\mathbf{0.928}\pm\mathbf{0.0}$ & $0.77\pm0.073$ & $0.677\pm0.057$ & $0.77\pm0.018$ & $0.854\pm0.022$ & $0.866\pm0.025$ \\
		Plane & $\mathbf{1.0}\pm\mathbf{0.0}$ & $\mathbf{1.0}\pm\mathbf{0.0}$ & $\mathbf{1.0}\pm\mathbf{0.0}$ & $\mathbf{1.0}\pm\mathbf{0.0}$ & $\mathbf{1.0}\pm\mathbf{0.0}$ & $\mathbf{1.0}\pm0.008$ \\
		PowerCons & $0.979\pm0.005$ & $0.984\pm0.006$ & $0.997\pm0.002$ & $\mathbf{1.0}\pm\mathbf{0.0}$ & $0.994\pm0.004$ & $\mathbf{1.0}\pm\mathbf{0.0}$ \\
		ProximalPhalanxOutlineAgeGroup & $0.858\pm\mathbf{0.002}$ & $0.842\pm0.007$ & $0.866\pm\mathbf{0.002}$ & $\mathbf{0.872}\pm\mathbf{0.002}$ & $0.869\pm0.005$ & $0.868\pm0.003$ \\
		ProximalPhalanxOutlineCorrect & $0.897\pm0.007$ & $0.895\pm0.011$ & $0.925\pm0.009$ & $0.918\pm\mathbf{0.006}$ & $0.925\pm0.008$ & $\mathbf{0.927}\pm0.007$ \\
		ProximalPhalanxTW & $0.814\pm\mathbf{0.004}$ & $0.778\pm0.018$ & $\mathbf{0.863}\pm0.006$ & $0.862\pm0.006$ & $0.861\pm0.005$ & $0.852\pm0.016$ \\
		RefrigerationDevices & $0.533\pm0.01$ & $0.498\pm0.009$ & $0.556\pm0.014$ & $0.565\pm0.008$ & $\mathbf{0.566}\pm\mathbf{0.007}$ & $0.529\pm0.01$ \\
		Rock & $0.68\pm\mathbf{0.0}$ & $0.744\pm0.039$ & $0.62\pm0.028$ & $0.704\pm0.067$ & $0.733\pm0.037$ & $\mathbf{0.768}\pm0.029$ \\
		ScreenType & $0.477\pm0.014$ & $0.604\pm0.008$ & $0.633\pm\mathbf{0.003}$ & $0.631\pm0.006$ & $0.635\pm0.007$ & $\mathbf{0.65}\pm0.011$ \\
		SemgHandGenderCh2 & $\mathbf{0.901}\pm\mathbf{0.005}$ & $0.878\pm0.009$ & $0.878\pm\mathbf{0.005}$ & $0.885\pm\mathbf{0.005}$ & $0.891\pm0.006$ & $0.88\pm0.006$ \\
		SemgHandMovementCh2 & $\mathbf{0.591}\pm0.006$ & $0.446\pm0.018$ & $0.451\pm\mathbf{0.002}$ & $0.462\pm0.012$ & $0.469\pm0.023$ & $0.461\pm0.03$ \\
		SemgHandSubjectCh2 & $\mathbf{0.843}\pm0.007$ & $0.652\pm0.025$ & $0.669\pm0.011$ & $0.693\pm\mathbf{0.006}$ & $0.695\pm0.016$ & $0.688\pm0.029$ \\
		ShapeletSim & $\mathbf{1.0}\pm\mathbf{0.0}$ & $0.712\pm0.03$ & $0.997\pm0.002$ & $0.967\pm0.041$ & $0.991\pm0.007$ & $\mathbf{1.0}\pm0.002$ \\
		ShapesAll & $0.91\pm\mathbf{0.001}$ & $0.911\pm0.003$ & $0.895\pm0.004$ & $0.916\pm0.002$ & $0.919\pm0.003$ & $\mathbf{0.922}\pm0.003$ \\
		SmallKitchenAppliances & $\mathbf{0.82}\pm0.006$ & $0.753\pm0.007$ & $0.814\pm\mathbf{0.004}$ & $0.817\pm0.007$ & $0.813\pm0.008$ & $0.803\pm0.008$ \\
		SmoothSubspace & $0.979\pm0.003$ & $0.974\pm0.006$ & $0.99\pm0.004$ & $0.993\pm0.004$ & $0.991\pm\mathbf{0.002}$ & $\mathbf{1.0}\pm0.008$ \\
		SonyAIBORobotSurface1 & $0.918\pm\mathbf{0.001}$ & $0.887\pm0.014$ & $0.916\pm0.026$ & $0.902\pm0.026$ & $0.903\pm0.028$ & $\mathbf{0.969}\pm0.027$ \\
		SonyAIBORobotSurface2 & $0.918\pm\mathbf{0.003}$ & $\mathbf{0.97}\pm0.006$ & $0.962\pm0.004$ & $0.961\pm\mathbf{0.003}$ & $0.964\pm\mathbf{0.003}$ & $0.953\pm0.012$ \\
		StarLightCurves & $\mathbf{0.981}\pm0.001$ & $0.977\pm0.001$ & $0.98\pm\mathbf{0.0}$ & $0.98\pm\mathbf{0.0}$ & $\mathbf{0.981}\pm\mathbf{0.0}$ & $\mathbf{0.981}\pm\mathbf{0.0}$ \\
		Strawberry & $0.981\pm0.002$ & $0.982\pm0.002$ & $0.983\pm0.002$ & $\mathbf{0.985}\pm\mathbf{0.001}$ & $0.983\pm0.002$ & $0.981\pm0.002$ \\
		SwedishLeaf & $0.964\pm\mathbf{0.002}$ & $0.968\pm0.005$ & $0.972\pm0.004$ & $0.974\pm0.004$ & $0.973\pm0.005$ & $\mathbf{0.975}\pm0.003$ \\
		Symbols & $0.974\pm\mathbf{0.001}$ & $0.981\pm0.003$ & $0.976\pm0.005$ & $0.983\pm\mathbf{0.001}$ & $\mathbf{0.984}\pm0.002$ & $0.975\pm0.004$ \\
		SyntheticControl & $0.999\pm0.002$ & $0.996\pm0.001$ & $0.998\pm0.001$ & $0.998\pm0.001$ & $0.997\pm\mathbf{0.0}$ & $\mathbf{1.0}\pm0.002$ \\
		ToeSegmentation1 & $0.968\pm\mathbf{0.002}$ & $0.973\pm0.006$ & $0.98\pm0.003$ & $0.97\pm0.006$ & $0.976\pm0.004$ & $\mathbf{0.988}\pm0.012$ \\
		ToeSegmentation2 & $0.942\pm0.006$ & $0.956\pm0.005$ & $0.962\pm\mathbf{0.002}$ & $0.967\pm0.01$ & $0.964\pm0.006$ & $\mathbf{0.969}\pm0.004$ \\
		Trace & $\mathbf{1.0}\pm\mathbf{0.0}$ & $\mathbf{1.0}\pm\mathbf{0.0}$ & $\mathbf{1.0}\pm\mathbf{0.0}$ & $\mathbf{1.0}\pm\mathbf{0.0}$ & $\mathbf{1.0}\pm\mathbf{0.0}$ & $\mathbf{1.0}\pm\mathbf{0.0}$ \\
		TwoLeadECG & $\mathbf{0.999}\pm\mathbf{0.0}$ & $0.997\pm0.002$ & $0.998\pm\mathbf{0.0}$ & $\mathbf{0.999}\pm\mathbf{0.0}$ & $0.998\pm0.001$ & $\mathbf{0.999}\pm\mathbf{0.0}$ \\
		TwoPatterns & $\mathbf{1.0}\pm\mathbf{0.0}$ & $\mathbf{1.0}\pm\mathbf{0.0}$ & $\mathbf{1.0}\pm\mathbf{0.0}$ & $\mathbf{1.0}\pm\mathbf{0.0}$ & $\mathbf{1.0}\pm\mathbf{0.0}$ & $\mathbf{1.0}\pm\mathbf{0.0}$ \\
		UMD & $0.986\pm\mathbf{0.0}$ & $\mathbf{0.995}\pm\mathbf{0.0}$ & $0.993\pm0.002$ & $\mathbf{0.995}\pm\mathbf{0.0}$ & $\mathbf{0.995}\pm\mathbf{0.0}$ & $\mathbf{0.995}\pm0.003$ \\
		UWaveGestureLibraryAll & $\mathbf{0.975}\pm\mathbf{0.001}$ & $0.894\pm0.002$ & $0.898\pm0.002$ & $0.897\pm0.002$ & $0.897\pm0.003$ & $0.899\pm0.002$ \\
		UWaveGestureLibraryX & $\mathbf{0.85}\pm\mathbf{0.001}$ & $0.805\pm0.004$ & $0.817\pm0.002$ & $0.816\pm0.002$ & $0.816\pm0.003$ & $0.819\pm0.002$ \\
		UWaveGestureLibraryY & $\mathbf{0.773}\pm\mathbf{0.002}$ & $0.728\pm0.005$ & $0.749\pm0.004$ & $0.741\pm0.003$ & $0.737\pm0.004$ & $0.744\pm0.005$ \\
		UWaveGestureLibraryZ & $\mathbf{0.792}\pm\mathbf{0.002}$ & $0.761\pm\mathbf{0.002}$ & $0.77\pm0.006$ & $0.767\pm0.003$ & $0.768\pm0.003$ & $0.77\pm0.003$ \\
		Wafer & $0.999\pm\mathbf{0.0}$ & $0.998\pm\mathbf{0.0}$ & $\mathbf{1.0}\pm\mathbf{0.0}$ & $0.999\pm\mathbf{0.0}$ & $0.999\pm\mathbf{0.0}$ & $0.998\pm0.001$ \\
		Wine & $0.804\pm\mathbf{0.022}$ & $0.774\pm0.025$ & $0.833\pm0.079$ & $0.811\pm0.156$ & $0.833\pm0.059$ & $\mathbf{0.926}\pm0.03$ \\
		WordSynonyms & $\mathbf{0.753}\pm\mathbf{0.002}$ & $0.692\pm0.01$ & $0.644\pm0.011$ & $0.707\pm0.005$ & $0.699\pm0.006$ & $0.707\pm0.008$ \\
		Worms & $0.743\pm\mathbf{0.01}$ & $0.754\pm0.03$ & $0.738\pm0.011$ & $0.81\pm0.02$ & $\mathbf{0.822}\pm0.018$ & $0.752\pm0.04$ \\
		WormsTwoClass & $0.79\pm0.015$ & $0.808\pm0.03$ & $0.891\pm0.011$ & $0.884\pm\mathbf{0.009}$ & $0.9\pm\mathbf{0.009}$ & $\mathbf{0.906}\pm0.029$ \\
		Yoga & $\mathbf{0.912}\pm0.005$ & $0.909\pm\mathbf{0.004}$ & $0.898\pm\mathbf{0.004}$ & $0.908\pm0.008$ & $0.909\pm0.005$ & $0.91\pm0.006$ \\
		\hline
		Total accuracy wins & $40$ & $11$ & $17$ & $30$ & $29$ & $54$ \\
		Total std wins & $76$ & $21$ & $35$ & $34$ & $27$ & $17$ \\
		\hline
	\end{tabular}
\end{table*}





% trigger a \newpage just before the given reference
% number - used to balance the columns on the last page
% adjust value as needed - may need to be readjusted if
% the document is modified later
%\IEEEtriggeratref{8}
% The "triggered" command can be changed if desired:
%\IEEEtriggercmd{\enlargethispage{-5in}}

% references section

\bibliographystyle{IEEEtran}
\bibliography{IEEEabrv,TNNLS}

\begin{IEEEbiography}[{\includegraphics[width=1in,height=1.25in,clip,keepaspectratio]{Figure/biography/gxy.png}}]{Xueyuan Gong}
is currently an assistant professor at the school of Intelligent Systems Science and Engineering, Jinan University, Zhuhai, China. He obtained his Ph.D. and M.Sc. degree in Computer Science from University of Macau in 2014 and 2019 respectively. Before that, he received his B.Sc. degree from Macau University of Science and Technology in 2011. His research interests include Machine Learning and Data Mining.
\end{IEEEbiography}

% insert where needed to balance the two columns on the last page with
% biographies
%\newpage

\begin{IEEEbiography}[{\includegraphics[width=1in,height=1.25in,clip,keepaspectratio]{Figure/biography/syw.jpg}}]{Yain-Whar Si}
is an associate professor at the University of Macau. He holds a Ph.D. degree in Information Technology from the Queensland University of Technology, Australia. His research interests are in the areas of Financial Technology, Computational Intelligence, and Information Visualization.
\end{IEEEbiography}

\begin{IEEEbiography}[{\includegraphics[width=1in,height=1.25in,clip,keepaspectratio]{Figure/biography/tyq.jpg}}]{Yongqi Tian}
received his B.S. degree from Beijing Institute of Technology, China, in 2019. He is currently pursuing a master’s degree in optical engineering at the Beijing Institute of Technology. His current research interests include Computer Vision and Deep Learning.
\end{IEEEbiography}

\begin{IEEEbiography}[{\includegraphics[width=1in,height=1.25in,clip,keepaspectratio]{Figure/biography/lc.jpg}}]{Cong Lin}
is currently an assistant professor at the School of Intelligent Systems Science and Engineering, Jinan University, Zhuhai, China. He obtained his Ph.D. degree in Software Engineering from University of Macau in 2017. His research interests include pattern recognition, machine learning, and computer vision. He is currently working on applicational image to image transforms using deep learning and vision-based object detectors for practical use.
\end{IEEEbiography}

\begin{IEEEbiography}[{\includegraphics[width=1in,height=1.25in,clip,keepaspectratio]{Figure/biography/zxy.jpg}}]{Xinyuan Zhang}
received the B.S. and Ph.D. degrees in computer science from Sun Yat-sen University, Guangzhou, China, in 2014 and 2019, respectively. From 2015 to 2017, he was a Research Assistant with the Department of Electronic Engineering, City University of Hong Kong, Hong Kong, China. His current research interests include evolutionary computation algorithms, swarm intelligence algorithms, large-scale optimization their applications in real-world problems.
\end{IEEEbiography}

\begin{IEEEbiography}[{\includegraphics[width=1in,height=1.25in,clip,keepaspectratio]{Figure/biography/lxx.jpg}}]{Xiaoxiang Liu}
is currently an associate professor at the School of Intelligent Systems Science and Engineering, Jinan University, Zhuhai, China. He obtained his Ph.D. degree, M.Sc. degree, and B.Sc. degree all in Aeronautical and Astronautical Manufacturing Engineering from Northwestern Polytechnical University. His research interests include Computer Vision and Artificial Intelligence.
\end{IEEEbiography}

% You can push biographies down or up by placing
% a \vfill before or after them. The appropriate
% use of \vfill depends on what kind of text is
% on the last page and whether or not the columns
% are being equalized.

%\vfill

% Can be used to pull up biographies so that the bottom of the last one
% is flush with the other column.
%\enlargethispage{-5in}

\end{document}
