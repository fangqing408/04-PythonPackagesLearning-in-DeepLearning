\documentclass[12pt,a4paper]{article}
\usepackage{ctex, hyperref}
\usepackage{amsmath, amsthm, amssymb, graphicx}
\usepackage{bm}
\usepackage{mathtools}
\usepackage{physics}
\title{Gradient Analysis of Standard Softmax and Normalized Softmax}
\author{fangqing}
\date{\today}
\setlength{\parindent}{0pt}
\begin{document}
\maketitle
\tableofcontents
\newpage
\section{Preface}
\subsection{前向传播}
\[
\text{feature: } \bm x\in\mathbb{R}^D,\quad
\text{target: } y\in\{1,2,\dots,C\},\quad
\text{dataset: } \{(\bm x^{(i)},y^{(i)})\}_{i=1}^N
\]
\[
\bm z=\begin{bmatrix}z_1\\z_2\\\vdots\\z_C\end{bmatrix}
=\underbrace{\begin{bmatrix}
w_{11}&w_{21}&\cdots&w_{D1}\\ 
w_{12}&w_{22}&\cdots&w_{D2}\\
\vdots&\vdots&\ddots&\vdots\\
w_{1C}&w_{2C}&\cdots&w_{DC}
\end{bmatrix}}_{\bm W^\top\in\mathbb{R}^{C\times D}}
\begin{bmatrix}x_1\\x_2\\\vdots\\x_D\end{bmatrix}
+\begin{bmatrix}b_1\\b_2\\\vdots\\b_C\end{bmatrix}
= \bm W^\top\bm x+\bm b
\]
\[
\hat{\bm y}=\mathrm{softmax}(\bm z)
=\begin{bmatrix}
e^{z_1}/\sum_{k=1}^C e^{z_k}\\
e^{z_2}/\sum_{k=1}^C e^{z_k}\\
\vdots\\e^{z_C}/\sum_{k=1}^C e^{z_k}
\end{bmatrix},
\qquad
L(\bm W,\bm b)=-\sum_{k=1}^C y_k\log \hat y_k
\]
\[
J(\bm W,\bm b)
=\frac{1}{N}\sum_{i=1}^N L^{(i)}
=-\frac{1}{N}\sum_{i=1}^N\sum_{k=1}^C y_k^{(i)}\log \hat y_k^{(i)}
\]
\subsection{反向传播}
\[
\frac{\partial J}{\partial w_{nm}}
=\sum_{i=1}^N \frac{\partial J}{\partial L^{(i)}}
\Big(\sum_{j=1}^C \frac{\partial L^{(i)}}{\partial \hat y_j^{(i)}}
\frac{\partial \hat y_j^{(i)}}{\partial z_n^{(i)}}\Big)
\frac{\partial z_n^{(i)}}{\partial w_{nm}}
\]
\[
J=\frac{1}{N}\sum_{i=1}^N L^{(i)},\quad
\frac{\partial J}{\partial L^{(i)}}=\frac{1}{N}
\]
\[
\frac{\partial L}{\partial \hat y_j}=-\frac{y_j}{\hat y_j},\qquad
\bm z=\bm W^\top\bm x+\bm b,\quad
z_n=\sum_{m=1}^D w_{mn}x_m + b_n,\quad
\frac{\partial z_n}{\partial w_{nm}}=x_m
\]
\section{Standard Softmax Gradient Analysis}
\subsection{雅可比矩阵}
\[
s=\sum_{k=1}^C e^{z_k},\qquad
\hat y_j=\frac{e^{z_j}}{s}
\]
对任意固定的类别 $\ell$,分别讨论两种情况:
\paragraph{(1) $\ell=j$:}
\[
\frac{\partial \hat y_j}{\partial z_j}
= \frac{\partial}{\partial z_j}\Big(\frac{e^{z_j}}{s}\Big)
= \frac{e^{z_j}}{s} - \frac{e^{z_j}}{s^2}\frac{\partial s}{\partial z_j}
= \hat y_j - \hat y_j\cdot \frac{e^{z_j}}{s}
= \hat y_j(1-\hat y_j)
\]
\paragraph{(2) $\ell\neq j$:}
\[
\frac{\partial \hat y_j}{\partial z_\ell}
= \frac{\partial}{\partial z_\ell}\Big(\frac{e^{z_j}}{s}\Big)
= -\,\frac{e^{z_j}}{s^2}\frac{\partial s}{\partial z_\ell}
= -\,\frac{e^{z_j}}{s^2}\,e^{z_\ell}
= -\,\hat y_j\,\hat y_\ell
\]
将上面两种情况统一写为:
\[
\boxed{\quad
\frac{\partial \hat y_j}{\partial z_\ell}
= \hat y_j\big(\mathbb 1_{j=\ell}-\hat y_\ell\big)
\quad}
\]
于是雅可比矩阵($C\times C$)为:
\[
\boxed{\quad
\frac{\partial \hat{\bm y}}{\partial \bm z}
=\mathrm{diag}(\hat{\bm y}) - \hat{\bm y}\,\hat{\bm y}^\top
\quad}
\]
\subsection{海森矩阵}
\[
\boxed{\quad
\frac{\partial L}{\partial z_j}
= \sum_{k=1}^C \frac{\partial L}{\partial \hat y_k}\frac{\partial \hat y_k}{\partial z_j}
= \sum_{k=1}^C \Big(-\frac{y_k}{\hat y_k}\Big)\,\hat y_k(\mathbb 1_{k=j}-\hat y_j)
= \hat y_j - \mathbb 1_{j=y}
\quad}
\]
\[
\boxed{
\nabla_{\bm z}^2 L
= \frac{\partial}{\partial \bm z}\big(\hat{\bm y}-\bm y\big)
= \underbrace{\frac{\partial \hat{\bm y}}{\partial \bm z}}_{\text{Jacobian of softmax}}
\;-\; \underbrace{\frac{\partial \bm y}{\partial \bm z}}_{=\,0}
= \mathrm{diag}(\hat{\bm y}) - \hat{\bm y}\,\hat{\bm y}^\top
}
\]
先看一维函数 $f(z)$,在 $z_0$ 附近:
\[
f(z_0+\Delta)\approx f(z_0)
+ f'(z_0)\,\Delta
+ \tfrac12\,f''(z_0)\,\Delta^2
\]
这里 $f'(z_0)$ 是斜率,$f''(z_0)$ 是弯曲程度
,多维函数 $f(\bm z)$,($\bm z\in\mathbb R^d$)方向导数定义为沿单位方向 $\bm v$ 的瞬时变化率:
\[
D_{\bm v}f(\bm z_0) \;=\; \lim_{t\to 0}\frac{f(\bm z_0+t\bm v)-f(\bm z_0)}{t}
\]
若 $f$ 可微,梯度 $\nabla f(\bm z_0)\in\mathbb R^d$ 满足
\[
\boxed{\,D_{\bm v} f(\bm z_0)=\nabla f(\bm z_0)\cdot \bm v\,}
\quad(\text{任意单位向量 }\bm v\text{,梯度在 } \bm v  \text{ 上的投影})
\]
特殊地,沿坐标轴方向 $\bm e_i$,$D_{\bm e_i}f=\partial f/\partial z_i$偏导数就是方向导数的特例,最陡上升方向正是 $\nabla f$ 本身,点积在 $\bm v$ 与 $\nabla f$ 同向时取最大

设二维标量函数
\[
f(x,y)=x^2+xy
\]
\[
\frac{\partial f}{\partial x}=2x+y,
\qquad
\frac{\partial f}{\partial y}=x
\]
\(\partial f/\partial x\) 是沿 \((1,0)\) 方向的瞬时变化率,\(\partial f/\partial y\) 是沿 \((0,1)\) 方向的瞬时变化率
\[
\nabla f(x,y)=\big(2x+y,\;x\big)
\]
\(\nabla f\) 指向函数上升最快的方向,\(\|\nabla f\|\) 等于最大上升速率,
对任意单位向量 \(\bm v=(v_x,v_y)\),方向导数定义为
\[
D_{\bm v} f(x,y)=\nabla f(x,y)\cdot \bm v
=(2x+y)\,v_x + x\,v_y.
\]
偏导是方向导数的特例:
\[
D_{(1,0)}f=\frac{\partial f}{\partial x},\qquad
D_{(0,1)}f=\frac{\partial f}{\partial y}.
\]
取点 \((x_0,y_0)=(2,1)\):
\[
\frac{\partial f}{\partial x}(2,1)=2\cdot2+1=5,\qquad
\frac{\partial f}{\partial y}(2,1)=2.
\]
\[
\nabla f(2,1)=(5,\,2).
\]
沿 45° 方向(单位向量 \(\bm v=\tfrac{1}{\sqrt{2}}(1,1)\))的方向导数:
\[
D_{\bm v}f(2,1)=\nabla f(2,1)\cdot \bm v
=(5,2)\cdot \frac{1}{\sqrt{2}}(1,1)
=\frac{7}{\sqrt{2}}\approx 4.95
\]
从 \((2,1)\) 朝 45° 方向迈一个单位小步,\(f\) 大约增加 \(4.95\),沿梯度方向,最陡上升方向的单位向量
\[
\hat{\bm g}=\frac{\nabla f(2,1)}{\|\nabla f(2,1)\|}
=\frac{(5,2)}{\sqrt{5^2+2^2}}
=\frac{(5,2)}{\sqrt{29}}
\]
其方向导数为
\[
D_{\hat{\bm g}}f(2,1)=\nabla f(2,1)\cdot \hat{\bm g}
=\|\nabla f(2,1)\|=\sqrt{29}\approx 5.385
\]
说明沿梯度方向的上升速率最大

从 \((2,1)\) 朝某单位方向 \(\bm v\) 走极小步长 \(t\):
\[
f(2,1+t\bm v)\;\approx\; f(2,1)\;+\; t\,D_{\bm v}f(2,1)
\;=\; f(2,1)\;+\; t\,\nabla f(2,1)\cdot \bm v
\]
例如取 \(\bm v=\tfrac{1}{\sqrt{2}}(1,1)\),有
\[
f(2,1+t\bm v)\approx f(2,1)+t\cdot \frac{7}{\sqrt{2}}
\]
若 $f$ 二次可微,海森矩阵(Hessian)$H(\bm z_0)=\nabla^2 f(\bm z_0)\in\mathbb R^{d\times d}$ 收集了所有二阶偏导。
沿方向 $\bm v$ 的二阶方向导数为,通过二次型 $\bm v^\top H\bm v$ 投影到某个方向
\[
\boxed{\,D_{\bm v}^2 f(\bm z_0)=\bm v^\top H(\bm z_0)\,\bm v\,}
\]
% Setup: g(t) = f(z0 + t v)
\newcommand{\vz}{\bm z_0}
\newcommand{\vv}{\bm v}
% First order
% D_v f(z0) = g'(0) = ∇f(z0) · v
从一维函数的角度看高维函数的一二阶方向导函数的证明:
\[
g(t)=f(\vz+t\vv)
\]
\[g'(t)=\nabla f(\vz+t\vv)\cdot \vv
\;\Rightarrow\;
D_{\vv}f(\vz)=g'(0)=\nabla f(\vz)\cdot \vv
\]
% Second order
% D_v^2 f(z0) = g''(0) = v^T H(z0) v
\[
g''(t)=\frac{d}{dt}\!\big(\nabla f(\vz+t\vv)\cdot \vv\big)
\]\[
= \big(\nabla^2 f(\vz+t\vv)\,\vv\big)\cdot \vv
\;\Rightarrow\;
\boxed{\,D_{\vv}^2 f(\vz)=\vv^\top \nabla^2 f(\vz)\,\vv\,}.
\]
海森矩阵的直观含义如下:
% Symmetric H: eigen-decomp, weighted energy, Rayleigh quotient (concise)
% Assumptions
% H ∈ R^{d×d} symmetric ⇒ H = QΛQ^T,  Q = [u_1 … u_d],  Λ = diag(λ_1,…,λ_d).
% Decompose v in eigenbasis: v = Σ_i α_i u_i,  with α = Q^T v,  Σ_i α_i^2 = ||v||^2.
% Quadratic form as weighted energy of principal curvatures
\[
v^\top H v
= v^\top Q\Lambda Q^\top v
= (Q^\top v)^\top \Lambda (Q^\top v)
= \sum_{i=1}^d \lambda_i \,\alpha_i^2.
\]
% Unit vector case: true weighted average of {λ_i}
\[
\|v\|=1 \;\Rightarrow\; \sum_i \alpha_i^2=1
\quad\Rightarrow\quad
v^\top H v \;=\; \sum_{i=1}^d \lambda_i \,\underbrace{\alpha_i^2}_{\text{weights}}.
\]
当 $\bm v$ 不是单位向量的时候:
% General v: Rayleigh quotient = normalized weighted average
\[
R_H(v)\;:=\;\frac{v^\top H v}{v^\top v}
=\frac{\sum_i \lambda_i \alpha_i^2}{\sum_i \alpha_i^2}
=\sum_{i=1}^d \lambda_i \,\underbrace{\frac{\alpha_i^2}{\sum_j \alpha_j^2}}_{\text{normalize weights}}
\]
\(\alpha_i^2\) 是 \(\bm v\) 在主方向 \(u_i\) 上的投影长度平方,归一化后各权重之和为 1,
因此 \(R_H(v)\) 就是把各主曲率 \(\lambda_i\) 按能量占比做加权平均
% 2) 极值性质与上下界
\[
\lambda_{\min}\;\le\; R_H(v)\;\le\;\lambda_{\max},\qquad
R_H(u_{\max})=\lambda_{\max},\;\;R_H(u_{\min})=\lambda_{\min}
\]
加权平均必落在最小最大特征值之间;当 \(\bm v\) 与 \(u_{\min}\)(或 \(u_{\max}\))对齐时,
全部权重集中在该主方向上,Rayleigh 商取到上下界

设 logits 向量 $\mathbf z=(z_1,\dots,z_C)^\top$,softmax 定义为
\[
\hat y_k(\mathbf z)\;=\;\frac{e^{z_k}}{\sum_{j=1}^C e^{z_j}},\qquad k=1,\dots,C
\]
对任意常数 $c\in\mathbb R$,有
\[
\hat y_k(\mathbf z+c\,\mathbf 1)
=\frac{e^{z_k+c}}{\sum_{j=1}^C e^{z_j+c}}
=\frac{e^{z_k}\,e^{c}}{e^{c}\sum_{j=1}^C e^{z_j}}
=\hat y_k(\mathbf z),
\]
即
\[
\mathrm{softmax}(\mathbf z+c\,\mathbf 1) \;=\; \mathrm{softmax}(\mathbf z)\quad(\forall c)
\]
令交叉熵损失
\[
L(\mathbf z)\;=\;-\sum_{k=1}^C y_k \log \hat y_k(\mathbf z)
\]
(其中 $\mathbf y$ 为目标分布,one-hot 时 $\sum_k y_k=1$)。由于 $L$ 只依赖 $\hat{\mathbf y}$,
\[
L(\mathbf z+c\,\mathbf 1)\;=\;L(\mathbf z)\quad(\forall c)
\]
因此沿着 $\mathbf 1$ 方向,$L$ 为常数,故
\[
\nabla_{\mathbf z} L(\mathbf z)^\top \mathbf 1 \;=\; 0,\qquad
H(\mathbf z)\,\mathbf 1 \;=\; \mathbf 0,\qquad
\mathbf 1^\top H(\mathbf z)\,\mathbf 1 \;=\; 0
\]
另外,已知
\[
\nabla_{\mathbf z} L(\mathbf z) \;=\; \hat{\mathbf y}(\mathbf z)-\mathbf y,
\]
于是
\[
\nabla_{\mathbf z} L(\mathbf z)^\top \mathbf 1
= \sum_{k=1}^C (\hat y_k - y_k)
= \underbrace{\sum_{k=1}^C \hat y_k}_{=\,1}
- \underbrace{\sum_{k=1}^C y_k}_{=\,1}
= 0
\]
对任意方向 $\mathbf v\in\mathbb R^C$,
\[
\mathbf v^\top H(\mathbf z)\,\mathbf v
= \sum_{k=1}^C \hat y_k v_k^2 \;-\; \Big(\sum_{k=1}^C \hat y_k v_k\Big)^2
= \mathrm{Var}_{k\sim \hat{\mathbf y}}(v_k)\;\ge\;0
\]
% Variance identity: no independence needed
\[
\mathrm{Var}(X)
=\mathbb{E}\big[(X-\mathbb{E}[X])^2\big]
=\mathbb{E}\big[X^2-2X\,\mathbb{E}[X]+\mathbb{E}[X]^2\big]
\]\[
=\mathbb{E}[X^2]-2\,\mathbb{E}[X]\mathbb{E}[X]+\mathbb{E}[X]^2
=\mathbb{E}[X^2]-\big(\mathbb{E}[X]\big)^2.
\]
因此 $H(\mathbf z)\succeq 0$,
当 $\mathbf v$ 为常数向量(如 $\mathbf v=\mathbf 1$)时,方差为 0,故该方向的二阶导为 0,
说明 $L$ 为凸但非严格凸,  
并且由于 $\sum_k \hat y_k=1$,取 $\bm v=\bm 1$ 有
\(
\bm v^\top H \bm v = \mathrm{Var}( \text{常数} )=0
\),
说明 $H$ 不是正定而是半正定
%========================
% Softmax 平移不变性与 Hessian 性质
%========================
\subsection{Standard Softmax Analysis}
由 $z_j=\bm w_j^\top\bm x+b_j$ 得
\[
\boxed{\quad
\pdv{L}{\bm x} = \sum_{j=1}^C (\hat y_j-\mathbb 1_{j=y})\,\bm w_j
= \underbrace{\sum_{j\neq y}\hat y_j \bm w_j}_{\text{负类排斥合力}}
- \underbrace{\bm w_y}_{\text{正类吸引}}\quad}
\]
真类系数是负数,把 $\bm w_y$ 拉向 $\bm x$,负类系数是正数,把 $\bm w_j$ 推离 $\bm x$,每看一个样本,就把正类中心拉向样本,把负类中心推离样本
\[
\boxed{\quad
\pdv{L}{\bm w_j}=(\hat y_j-\mathbb 1_{j=y})\,\bm x,\qquad
\pdv{L}{b_j}=\hat y_j-\mathbb 1_{j=y}\quad}
\]
范数上界,把 W 看成一个线性放大器,这个最大放大倍数就叫谱范数也等于最大奇异值:
\[
\hat{\bm y}\approx \bm e_k\ (k\neq y)
\;\Rightarrow\;
\hat{\bm y}-\bm e_y \approx \bm e_k-\bm e_y,\qquad
\|\bm e_k-\bm e_y\|_2=\sqrt{1^2+(-1)^2}=\sqrt{2}.
\]
\[
\Big\|\pdv{L}{\bm x}\Big\|
= \|W(\hat{\bm y}-\bm e_y)\|
\le \|W\|_2\,\|\hat{\bm y}-\bm e_y\|_2
\le \sqrt2\,\|W\|_2,
\qquad
\Big\|\pdv{L}{\bm w_j}\Big\|\le \|\bm x\|.
\]
在标准形式 \(z_j=\mathbf w_j^\top \mathbf x + b_j\) 下,若缺少归一化/正则,训练倾向通过增大
\(\|W\|\)(有时也增大 \(\|\mathbf x\|\))来放大 \(z_y - z_{k\neq y}\),而非优化角度;这会提升
\(\|W\|_2\) 最大放大倍数,谱范数,使
\[
\big\|\tfrac{\partial L}{\partial \mathbf x}\big\|
=\|W(\hat{\mathbf y}-\mathbf e_y)\|
\;\le\; \|W\|_2\,\|\hat{\mathbf y}-\mathbf e_y\|
\;\le\; \sqrt{2}\,\|W\|_2
\]
的上界变大,链式反传更易爆/抖,若未做 \(\mathrm{log\text{-}sum\text{-}exp}\) 等数值稳定,
过大 logits 也会造成前向溢出,结果是模型更依赖强度(模长)而非方向(角度/判别边界),
判别边界变差,泛化与鲁棒性下降
% Stable log-sum-exp (LSE trick)
\[
\log\!\sum_{i} e^{z_i}
\;=\;
m \;+\; \log\!\sum_{i} e^{\,z_i-m}
\qquad(\text{任取 } m).
\]
取 \(m=\max_i z_i\) 时,\(z_i-m\le 0\)(所有指数 \(\le 1\)),数值稳定:
\[
\boxed{\;
\log\!\sum_{i} e^{z_i}
\;=\;
\max_i z_i
\;+\; \log\!\sum_{i} e^{\,z_i-\max_j z_j}
\;}
\]
% =========================================================
% 3.(可选)Normalized / Cosine Softmax(简要保留)
% =========================================================
\section{Normalized Softmax Gradient Analysis}
定义
\[
\hat{\bm x}=\frac{\bm x}{\|\bm x\|},\quad
\hat{\bm w}_j=\frac{\bm w_j}{\|\bm w_j\|},\quad
z_j=s\,\hat{\bm w}_j^\top \hat{\bm x}=s\cos\theta_j,\quad z_j\in[-s,s]
\]
梯度对 $\bm z$ 的形式不变,但通过
\[
\pdv{\hat{\bm x}}{\bm x}
=\frac{1}{\|\bm x\|}\big(I-\hat{\bm x}\hat{\bm x}^\top\big),\qquad
\pdv{\hat{\bm w}_j}{\bm w_j}
=\frac{1}{\|\bm w_j\|}\big(I-\hat{\bm w}_j\hat{\bm w}_j^\top\big)
\]
% —— 归一化向量的微分 + 径向/切向敏感性 +(含文字分析与 d(x^T x) 展开)——
% 设 x∈R^d, x≠0,定义单位向量 \hat{x}=x/\|x\|。目标:推导 d\|x\| 与 d\hat{x},
% 并分析“径向/切向”扰动对长度与方向的一阶影响;最后给出归一化 softmax 的链式法则。
% 0) 直观说明
% \|x\| 是“长度”,\hat{x} 是“方向”。对很小的扰动 dx:
% - 径向分量(沿 \hat{x})只改变长度的一阶项,对方向的一阶影响为 0;
% - 切向分量(与 \hat{x} 正交)只改变方向的一阶项,对长度的一阶影响为 0;
% - \|x\| 越大,同样大小的切向扰动对方向的效果越小(1/\|x\| 缩放)。
% 1) 范数的微分:从 \|x\|=(x^\top x)^{1/2} 出发,完整展开 d(x^\top x)
\[
\|x\|=(x^\top x)^{1/2}.
\quad\text{因此}\quad
d\|x\|=\frac{1}{2}(x^\top x)^{-1/2}\,d(x^\top x)
\]
\[
\text{而 } 
d(x^\top x)=d\!\Big(\sum_{i=1}^d x_i^2\Big)
=\sum_{i=1}^d 2x_i\,dx_i
=2\sum_{i=1}^d x_i\,dx_i
=\boxed{\,2\,x^\top dx\,}
\]
代回可得
\[
d\|x\|
=\frac{1}{2}\cdot\frac{1}{\|x\|}\cdot 2\,x^\top dx
=\frac{x^\top dx}{\|x\|}
=\boxed{\,\hat{x}^\top dx\,}
\]
% 2) 归一化向量的微分:\hat{x}=x/\|x\|
\[
d\hat{x}
= d\!\Big(\frac{x}{\|x\|}\Big)
= \frac{dx}{\|x\|} - x\,\frac{d\|x\|}{\|x\|^2}
= \frac{1}{\|x\|}\Big(dx - \hat{x}\,\hat{x}^\top dx\Big)\]\[
=\boxed{\,\frac{1}{\|x\|}\big(I-\hat{x}\hat{x}^\top\big)\,dx\,}
\]
% —— 单位球面的切空间与正交投影(直观版)——
% 设 \hat{x}\in\mathbb{R}^d,且 \|\hat{x}\|=1。我们在单位球面 S^{d-1} 上考察方向 \hat{x}。
% 1) 切空间的定义与直觉
% 在点 \hat{x} 处,切空间是“贴着球面、与半径垂直”的超平面:
切空间的定义
\[
T_{\hat{x}}S^{d-1}
=\big\{\,v\in\mathbb{R}^d:\ \hat{x}^\top v=0\,\big\}
\]
球面上的某点,沿球表的方向就是切空间,指向地心/离心的是径向
% 2) 正交投影矩阵 P=I-\hat{x}\hat{x}^\top 的作用
% 将任意向量 v 分解为“径向分量 + 切向分量”:
\[
v = v_{\parallel}+v_{\perp},\qquad
v_{\parallel}=(\hat{x}^\top v)\,\hat{x}\quad(\text{沿半径}),\qquad
v_{\perp}=v-(\hat{x}^\top v)\,\hat{x}\quad(\text{贴着球面})
\]
把 v 丢进
\[
P:=I-\hat{x}\hat{x}^\top
\]
得到
\[
Pv=(I-\hat{x}\hat{x}^\top)v
= v - (\hat{x}^\top v)\,\hat{x}
= v_{\perp}
\]
\(P\) 会去掉径向分量 \(v_{\parallel}\),仅保留切向分量 \(v_{\perp}\)
% 3) 三个标准性质(正交投影应有的特性)
\[
P\hat{x}=(I-\hat{x}\hat{x}^\top)\hat{x}
=\hat{x}-(\hat{x}^\top\hat{x})\hat{x}=0
\]
该式子表示半径方向被完全杀掉,与 \(T_{\hat{x}}S^{d-1}=\hat{x}^\perp\) 一致
% 4) 和归一化向量微分的关系(备查)
% 若 \hat{x}=x/\|x\|,则
\[
d\hat{x}
=\frac{1}{\|x\|}\big(I-\hat{x}\hat{x}^\top\big)\,dx
=\frac{1}{\|x\|}P\,dx
\]
上面式子告诉我们,对于 Normalized Softmax,方向的变化 $d\hat{x}$ 来自切分量
% 3) 径向/切向敏感性:把 dx 分解为平行与正交分量
径向分量控制长度的一阶变化(这里不存在,被分解掉了);切向分量控制方向的一阶变化;
且方向变化幅度与 \(1/\|x\|\) 成正比 \(\|x\|\) 越大,方向越不敏感
% 4) 归一化 Softmax 的链式法则(备查)
% 设 \hat{w}_j=w_j/\|w_j\|,\ \hat{x}=x/\|x\|,\ z_j=s\,\hat{w}_j^\top\hat{x},\
% p_j=\mathrm{softmax}_j(z),\ \partial L/\partial z_j=p_j-\mathbb{1}_{j=y}。
\[
\frac{\partial L}{\partial x}
=\sum_j \frac{\partial L}{\partial z_j}\frac{\partial z_j}{\partial x}
=\sum_j (p_j-\mathbb{1}_{j=y})\; s\;\frac{1}{\|x\|}\big(I-\hat{x}\hat{x}^\top\big)\,\hat{w}_j
\]\[
=\frac{s}{\|x\|}\big(I-\hat{x}\hat{x}^\top\big)\sum_j (p_j-\mathbb{1}_{j=y})\,\hat{w}_j
\]
\[
\frac{\partial L}{\partial w_j}
=(p_j-\mathbb{1}_{j=y})\; s\;\frac{1}{\|w_j\|}\big(I-\hat{w}_j\hat{w}_j^\top\big)\,\hat{x}
\]
两个投影算子把更新限制在各自的切空间内,只改角度不鼓励改模长,
logits 有界于 \([-s,s]\) 提升数值稳定,系数 \(s/\|\cdot\|\) 决定更新尺度,
\(\|x\|\) 或 \(\|w_j\|\) 越大,单位变化对方向的影响越小,也就更加稳定
\section{PGA}
特征与类别权重均作单位化:
\[
x\in\mathbb{S}^{d-1},\quad
\hat w_j=\frac{w_j}{\|w_j\|}\in\mathbb{S}^{d-1},\quad
z_j=s\,\hat w_j^\top x,\quad
p_j=\mathrm{softmax}_j(z).
\]
单位球面在 $x$ 处的切空间与正交投影:
\[
T_x\mathbb{S}^{d-1}=\{v\in\mathbb{R}^d: x^\top v=0\},\qquad
\Pi_x:=I-xx^\top.
\]
注意对任意由 logits 组合得到的损失 $L(\{z_j\})$,链式法则给出:
\[
\frac{\partial L}{\partial x}
=\sum_j \frac{\partial L}{\partial z_j}\frac{\partial z_j}{\partial x}
=\sum_j (p_j-\mathbb{1}_{j=y})\; s\;\frac{1}{\|x\|}\big(I-\hat{x}\hat{x}^\top\big)\,\hat{w}_j
\]\[
=\frac{s}{\|x\|}\Pi_x \sum_j (p_j-\mathbb{1}_{j=y})\,\hat{w}_j
\]
\subsection{ArcFace 的切向合力}
\[
\hat x=\frac{x}{\|x\|},\quad \hat w_j=\frac{w_j}{\|w_j\|},\quad
z_y=s\cos(\theta_y+m),\quad z_j=s\cos\theta_j\ (j\ne y)
\]
\[
L=-\log\frac{e^{z_y}}{\sum_k e^{z_k}},\qquad
\frac{\partial L}{\partial z_j}=p_j-\mathbb 1_{j=y},\quad
p_j=\frac{e^{z_j}}{\sum_k e^{z_k}}
\]
% 切空间投影与“最快减角方向”
% 设 \hat x = x/\|x\|, \hat w_y = w_y/\|w_y\|, \hat w_y^\top \hat x = \cos\theta_y,
% 投影算子 \Pi_x = I - \hat x \hat x^\top.
% (2) 分解:径向 + 切向
\[
\hat w_y
= (\hat w_y^\top \hat x)\,\hat x
+ \big(\hat w_y - (\hat w_y^\top \hat x)\,\hat x\big)
= \cos\theta_y\,\hat x + \Pi_x\,\hat w_y
\]
% (3) 切向分量的长度
\[
\|\Pi_x \hat w_y\|^2
= \|\hat w_y - \cos\theta_y\,\hat x\|^2
= 1 + \cos^2\theta_y - 2\cos\theta_y(\hat w_y^\top \hat x)
= 1 - \cos^2\theta_y
= \sin^2\theta_y
\]
\[
\Rightarrow\quad
\|\Pi_x \hat w_y\|=\sin\theta_y,
\qquad
\Pi_x \hat w_y=\sin\theta_y\,t_y,
\quad
t_y:=\frac{\Pi_x \hat w_y}{\|\Pi_x \hat w_y\|}
\]
% 关键微分:角度与余弦对 x 的梯度
\[
\frac{\partial \cos\theta_j}{\partial x}=\frac{1}{\|x\|}\,\Pi_x\,\hat w_j,
\qquad
\frac{\partial \theta_y}{\partial x}=-\frac{1}{\|x\|}\,t_y
\]
\[
\cos\theta_j=\hat w_j^\top \hat x,\ \ 
\frac{\partial \hat x}{\partial x}
=\frac{1}{\|x\|}\big(I-\hat x\hat x^\top\big)=\frac{1}{\|x\|}\Pi_x\]\[
\Rightarrow
\frac{\partial \cos\theta_j}{\partial x}
=\Big(\frac{\partial \hat x}{\partial x}\Big)^\top \hat w_j
=\frac{1}{\|x\|}\Pi_x\hat w_j
\]
% ===== 式子 2:∂ θ_y / ∂ x =====
\[
\theta_y=\arccos(\hat w_y^\top \hat x)
\Rightarrow
\frac{\partial \theta_y}{\partial x}
=\frac{d\,\arccos(u)}{du}\bigg|_{u=\cos\theta_y}\cdot
\frac{\partial \cos\theta_y}{\partial x}\]\[
= -\frac{1}{\sin\theta_y}\cdot \frac{1}{\|x\|}\Pi_x\hat w_y
\]
又因为
\(
\sin\theta_y=\sqrt{1-(\hat w_y^\top \hat x)^2}
= \|\Pi_x\hat w_y\|
\),
故:
\[
\frac{\partial \theta_y}{\partial x}
=-\frac{1}{\|\Pi_x\hat w_y\|}\cdot \frac{1}{\|x\|}\Pi_x\hat w_y
=-\frac{1}{\|x\|}\,\frac{\Pi_x\hat w_y}{\|\Pi_x\hat w_y\|}
=-\frac{1}{\|x\|}\,t_y
\]
% 真类(带 margin)logit 的链式法则
\[
\frac{\partial z_y}{\partial x}
=s\frac{\partial \cos(\theta_y+m)}{\partial x}
=s\big(-\sin(\theta_y+m)\big)\frac{\partial \theta_y}{\partial x}
=\frac{s}{\|x\|}\,\sin(\theta_y+m)\,t_y
\]
将 \(t_y\) 换回 \(\Pi_x\hat w_y\) 得到“共线缩放”:
\[
\boxed{\;
\frac{\partial z_y}{\partial x}
=\frac{s}{\|x\|}\,\underbrace{\frac{\sin(\theta_y+m)}{\sin\theta_y}}_{:=\alpha(\theta_y,m)}
\,\Pi_x\hat w_y
\;}
\]
可见方向与 \(\Pi_x\hat w_y\) 共线,仅被 \(\alpha(\theta_y,m)\) 缩放;当 \(m=0\) 时 \(\alpha=1\)
% 负类(无 margin)logit
\[
\frac{\partial z_j}{\partial x}
=s\frac{\partial \cos\theta_j}{\partial x}
=\frac{s}{\|x\|}\,\Pi_x\,\hat w_j,\qquad j\ne y
\]
% 汇总得到 ArcFace 对 x 的梯度
\[
\begin{aligned}
\frac{\partial L}{\partial x}
&=\sum_j \frac{\partial L}{\partial z_j}\frac{\partial z_j}{\partial x}\\
&=(p_y-1)\frac{s}{\|x\|}\alpha(\theta_y,m)\Pi_x\hat w_y
\;+\;\sum_{j\ne y} p_j\,\frac{s}{\|x\|}\Pi_x\hat w_j\\
&=\frac{s}{\|x\|}\,\Pi_x
\Big(\sum_{j\ne y}p_j\,\hat w_j - \alpha(\theta_y,m)\,\hat w_y\Big)
\end{aligned}
\]
梯度完全落在 \(x\) 的切空间内(被 \(\Pi_x\) 投影),
“真类吸引”与“切向方向”严格共线,margin 仅通过 \(\alpha(\theta_y,m)\) 调整同一方向的力度
\subsection{PGA K/Z 对齐的切向合力}
对同类且可靠的邻居(经掩码与 EMA 平滑)集合 $\mathcal N_y(x)$,K/Z 对齐的一个统一写法是最大化方向一致性(或最小化某相似度损失):
\[
L_{\mathrm{pga}}
=\sum_{a\in\mathcal N_y(x)} \beta_a\,\phi(\langle x,\ x_a^{\mathrm{tgt}}\rangle),
\qquad \beta_a\ge 0,\ \ \phi'(\cdot)\le 0
\]
其中 $x_a^{\mathrm{tgt}}$ 来自下层/EMA 的目标方向,于是
\[
\pdv{L_{\mathrm{pga}}}{x}
=\sum_a \beta_a\,\phi'(\langle x,x_a^{\mathrm{tgt}}\rangle)\,x_a^{\mathrm{tgt}}
= -\sum_a \alpha_a\,x_a^{\mathrm{tgt}}\quad(\alpha_a:=\beta_a|\phi'|\ge 0)
\]
投影到切空间得
\begin{equation}
\label{eq:pga-grad}
-g_{\mathrm{pga}}
\ :=\ -\Pi_x\pdv{L_{\mathrm{pga}}}{x}
\ =\ \sum_{a\in\mathcal N_y(x)} \alpha_a\,\Pi_x x_a^{\mathrm{tgt}}
\end{equation}
即 PGA 的下降方向是同类目标邻居在切空间的加权均值

\subsection{是否对抗的充要判据(切空间内积)}
ArcFace 与 PGA 的是否“冲突”取决于它们在切空间的内积:
\begin{equation}
\label{eq:ip-test}
\left\langle -g_{\mathrm{arc}},\ -g_{\mathrm{pga}}\right\rangle
=\Big\langle \Pi_x u,\ \sum_{a} \alpha_a\,\Pi_x x_a^{\mathrm{tgt}}\Big\rangle
\end{equation}
\noindent 若该内积 $\ge 0$,两者同向/弱同向,不对抗;若 $<0$,则局部相悖,令
\[
\theta_a:=\angle\!\big(\Pi_x u,\ \Pi_x x_a^{\mathrm{tgt}}\big),\qquad
c_a:=\cos\theta_a
\]
由柯西不等式,
\[
\Big\langle \Pi_x u,\ \sum_a \alpha_a\,\Pi_x x_a^{\mathrm{tgt}}\Big\rangle
=\|\Pi_x u\|\sum_a \alpha_a\,\|\Pi_x x_a^{\mathrm{tgt}}\|\,c_a\]\[
\ \ge\
\|\Pi_x u\|\Big(\sum_a \alpha_a\,\|\Pi_x x_a^{\mathrm{tgt}}\|\Big)\cdot \min_a c_a
\]
因此,只要
\begin{equation}
\label{eq:sufficient}
\boxed{\ \min_{a\in\mathcal N_y(x)} \angle\!\big(\Pi_x u,\ \Pi_x x_a^{\mathrm{tgt}}\big)\ \le\ 90^\circ\ }
\end{equation}
就有 $\langle -g_{\mathrm{arc}},-g_{\mathrm{pga}}\rangle\ge 0$,充分保证两者不对抗,
这与实践中同类掩码加上可靠边相吻合,稳定邻居与真类方向通常形成锐角
\end{document}
